\begin{abstract}
\addchaptertocentry{\abstractname} % Add the abstract to the table of contents

Aflatoxins, a group of mycotoxins produced by various mold species within the genus \textit{Aspergillus}, have been extensively investigated for their potential to contaminate food and feed, rendering them unfit for consumption. Nevertheless, the role of aflatoxins as environmental contaminants in soil, which represents their natural habitat, remains a relatively unexplored area in aflatoxin research. This knowledge gap can be attributed, in part, to the methodological challenges associated with detecting aflatoxins in soil. The main objective of this PhD project was to develop and validate an analytical method that allows monitoring of aflatoxins in soil, and scrutinize the mechanisms and extent of occurrence of aflatoxins in soil, the processes governing their dissipation, and their impact on the soil microbiome and associated soil functions. By utilizing an efficient extraction solvent mixture comprising acetonitrile and water, coupled with an ultrasonication step, recoveries of 78\% to 92\% were achieved, enabling reliable determination of trace levels in soil ranging from 0.5 to 20 \textmu g kg\textsuperscript{-1}. However, in a field trial conducted in a high-risk model
region for aflatoxin contamination in Sub-Saharan Africa, no aflatoxins were detected using this procedure, underscoring the complexities of field monitoring. These challenges encompassed rapid degradation, spatial heterogeneity, and seasonal fluctuations in aflatoxin occurrence. Degradation experiments revealed the importance of microbial and photochemical processes in the dissipation of aflatoxins in soil with half-lives of 20 - 65 days. The rate of dissipation was found to be influenced by soil properties, most notably soil texture and the initial concentration of aflatoxins in the soil. An exposure study provided evidence that aflatoxins do not pose a substantial threat to the soil microbiome, encompassing microbial biomass, activity, and catabolic functionality. This was particularly evident in clayey soils, where the toxicity of aflatoxins diminished significantly due to their strong binding to clay minerals. However, several critical questions remain unanswered, emphasizing the necessity for further research to attain a more comprehensive understanding of the ecological importance of aflatoxins. Future research should prioritize the challenges associated with field monitoring of aflatoxins, elucidate the mechanisms responsible for the dissipation of aflatoxins in soil during microbial and photochemical degradation, and investigate the ecological consequences of aflatoxins in regions heavily affected by aflatoxins, taking into account the interactions between aflatoxins and environmental and anthropogenic stressors. Addressing these questions contributes to a comprehensive understanding of the environmental impact of aflatoxins in soil, ultimately contributing to more effective strategies for aflatoxin management in agriculture.


\end{abstract}