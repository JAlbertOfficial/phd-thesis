%========================================================================================
\section{Supporting Information on Chapter 1} \label{Annex_chap1}
%========================================================================================

%\clearpage

%----------------------------------------------------------------------------------------
%\subsection*{Aflatoxin levels in food commodities} 
%----------------------------------------------------------------------------------------

%++++++++++++++++++++++++++++++++++++++++++++++++++++++++++++++++++++++++++++++++++++++++
\begin{landscape}
\begingroup\footnotesize
\begin{longtable}[c]{llllllll}
\captionsetup{labelfont=bf, justification=justified, singlelinecheck=false, width=1.4\textwidth} 
\caption{Occurrence of AFB1, AFM1 and the sum of the aflatoxins (AFB1 + AFB2 + AFG1 + AFG2) in food products from countries across different geographical regions.} 
\\
\label{table:Aflatoxin_food_levels}
\\
\hline
\textbf{Geographical region} &
  \textbf{Country} &
  \textbf{Food product} &
  \textbf{Aflatoxin} &
  \textbf{N} &
  \textbf{\% pos.} &
  \textbf{Range \textsuperscript{[a]}} &
  \textbf{References} \\ \hline
\endfirsthead
%
\multicolumn{8}{c}%
{{\bfseries Table \thetable\ continued from previous page}} \\
\hline
\textbf{Geographical region} &
  \textbf{Country} &
  \textbf{Food product} &
  \textbf{Aflatoxin} &
  \textbf{Sample size} &
  \textbf{\% positive} &
  \textbf{Range \textsuperscript{[a]}} &
  \textbf{References} \\ \hline
\endhead
%
\hline
\endfoot
%
\endlastfoot
%
Central America & Costa Rica   & Milk                    & AFM1       & 70   & 95.7       & 0.019–0.629   & \citet{chavarria2015detection}       \\
                &              & Cheese                  &            & 70   & 37.2       & 0.031-0.276   &                                      \\
                & Haiti        & Peanut                  & \textSigma AFs & 8    & 25         & 186.6–375.1   & \citet{aristil2020fungal}            \\
                & Mexico       & Infant formula          & AFM1       & 55   & 20         & 0.040–0.450   & \citet{quevedo2020aflatoxin}         \\ \hline
East Asia       & China        & Cow milk                & AFM1       & 5650 & 4.7        & 0.0085–0.412  & \citet{li2017aflatoxin}              \\
                & China        & Cow milk                & AFM1       & 233  & 48.1       & 0.005–0.096   & \citet{guo2013aflatoxin}             \\
                &              & Yoghurt                 &            & 178  & 4.5        & 0.005–0.083   &                                      \\
                & China        & Peanuts                 & \textSigma AFs & 2494 & 0.15       & 0.06–1602.5   & \citet{wu2016aflatoxin}              \\
                & China        & Raw milk                & AFM1       & 1207 & 4.64       & 0.005–0.06    & \citet{li2018occurrence}             \\
                & China        & Rice                    & AFB1       & 370  & 63.5       & 0.03–20       & \citet{lai2015occurrence}            \\
                & China        & Wheat \& wheat crackers & AFB1       & 178  & 5.6        & 0.03–0.12     & \citet{zhao2018aflatoxin}            \\
                & Korea        & Functional foods        & AFB1       & 185  & 0          & ND            & \citet{lee2015analysis}              \\
                & Korea        & Brown rice              & AFB1       & 507  & 1          & 0.3–1.1       & \citet{kim2017simultaneous}          \\
                &              & Millet                  &            &      & 9          & 0.4–5.6       &                                      \\
                &              & Sorghum                 &            &      & 4          & 0.7–1.7       &                                      \\
                &              & Maize                   &            &      & 1          & 0.7–5.2       &                                      \\
                &              & Mixed cereals           &            &      & 4          & 0.4–12.4      &                                      \\
                & Korea        & Soybean paste           & \textSigma AFs & 45   & 24.4       & 0.88–16.17    & \citet{jeong2019natural}             \\
                & Taiwan       & Peanut products         & \textSigma AFs & 1827 & 32.7       & 0.2–513.4     & \citet{chen2013survey}               \\
                & Taiwan       & Peanuts                 & AFB1       & 1089 & 25         & 0.2–432       & \citet{lien2019assessing}            \\
                &              &                         & AFB2       &      &            & 0.1–130.9     &                                      \\
                &              &                         & AFG1       &      &            & 0.2–113       &                                      \\
                &              &                         & AFG2       &      &            & 0.1–17        &                                      \\
                &              &                         & \textSigma AFs &      &            & 0.1–441       &                                      \\ \hline
Middle East/    & Egypt        & Maize                   & AFB1       & 61   & 25         & 0.02–44.9     & \citet{f2019mycotoxin}               \\
North Africa    &              &                         & AFB2       & 10   & 0.1-7.0    &               &                                      \\
                &              & Milk                    & AFM1       & 20   & 0.02-0.19  &               &                                      \\
                & Egypt        & Meat products           & \textSigma AFs & 50   & 100        & 0.47–2.1      & \citet{abd2015rapid}                 \\
                & Egypt        & Wheat                   & AFB1       & 36   & 33.3       & 0.04–62.17    & \citet{hathout2020incidence}         \\
                &              &                         & AFB2       & 75   & 0.12-3.82  &               &                                      \\
                &              &                         & AFG1       & 100  & 0.09-48.59 &               &                                      \\
                &              &                         & AFG2       & 100  & 0.11-10.93 &               &                                      \\
                & Iran         & Cow milk                & AFM1       & 64   & 84.4       & 0.006–0.188   & \citet{bahrami2016aflatoxin}         \\
                &              & Yoghurt                 &            & 42   & 23.8       & 0.006–0.021   &                                      \\
                & Iran         & Rice                    & AFB1       & 40   & 100        & 0.29–2.92     & \citet{eslami2015determination}      \\
                & Iran         & Wheat flour             & \textSigma AFs & 180  & 80         & 0.01–0.5      & \citet{jahanbakhsh2021probabilistic} \\
                & Lebanon      & Infant formula          & AFM1       & 84   & 88         & 0.005–0.0481  & \citet{elaridi2019aflatoxin}         \\
                & Lebanon      & Spices                  & AFB1       & 94   & 16         & 2.2–1118.3    & \citet{el2019multimycotoxins}        \\
                &              & Herbs                   &            & 38   & 8          & 8.7–62.7      &                                      \\
                & Morocco      & Tea                     & \textSigma AFs & 1290 & 58.9       & 1.2–116.2     & \citet{mannani2020assessment}        \\
                & Saudi Arabia & Nuts                    & \textSigma AFs & 264  & 26.5       & 1.0–110       & \citet{neamatallah2013incidence}     \\
                & Tunisia      & Pearl Millet            & AFB1       & 220  & 8.6        & 0.24–1046     & \citet{houissa2019multimycotoxin}    \\
                &              &                         & AFB2       &      & 0.5        & 0.4–96.1      &                                      \\
                &              &                         & AFG1       &      & 0.5        & 0.32–20.3     &                                      \\
                &              &                         & AFM1       &      & 0.5        & 0.4–18.1      &                                      \\
                & Yemen        & Roasted coffee beans    & \textSigma AFs & 25   & 100        & 14.255–23.231 & \citet{humaid2019aflatoxins}         \\
                &              & Green coffee beans      &            & 25   & 100        & 14.694–27.176 &                                      \\ \hline
North America   & USA          & Chilies                 & AFB1       & 169  & 63.9       & 2–94.9        & \citet{singh2017aflatoxin}           \\ \hline
South America   & Brazil       & Cashew nuts             & \textSigma AFs & 70   & 34.3       & 0.60–31.5     & \citet{milhome2014occurrence}        \\
                & Brazil       & Cocoa beans             & \textSigma AFs & 123  & 16.3       & 0.35–30       & \citet{pires2019aflatoxins}          \\
                & Brazil       & Cow milk                & AFM1       & 129  & 14         & 0.0002–0.1057 & \citet{picinin2013influence}         \\
                & Brazil       & Maize                   & \textSigma AFs & 148  & 38         & 0.5–49.9      & \citet{oliveira2017natural}          \\
                & Brazil       & Peanuts                 & \textSigma AFs & 119  & 10         & 0.3–100       & \citet{martins2017biodiversity}      \\
                & Colombia     & Maize                   & AFB1       & 20   & 15         & 6.4–458.2     & \citet{diaz2015mycotoxins}           \\
                &              &                         & AFB2       &      & 15         & 1.9–55.5      &                                      \\
                &              &                         & AFG1       &      & 5          & 72.2–72.2     &                                      \\
                & Peru         & Maize                   & \textSigma AFs & 82   & 64.6       & 1–17          & \citet{coloma2019mycotoxin}          \\
                & Uruguay      & Sorghum                 & AFB1       & 275  & 0.7        & 1–14          & \citet{del2016evolution}             \\ \hline
South Asia      & India        & Corn                    & AFB1       & 150  & 100        & 48–383        & \citet{mudili2014mould}              \\
                & India        & Rice                    & \textSigma AFs & 87   & 2.3        & 21.581–22.989 & \citet{mukherjee2019study}           \\
                & India        & Sorghum                 & AFB1       & 15   & 71.4       & 0.005–0.02    & \citet{jayashree2019effect}          \\
                & India        & Spices                  & \textSigma AFs & 55   & 85.4       & 4–219.6       & \citet{jeswal2015mycobiota}          \\
                & Pakistan     & Dates \& dates products & \textSigma AFs & 57   & 31.6       & 0.15–16.70    & \citet{iqbal2014aflatoxins}          \\
                & Pakistan     & Milk                    & AFM 1      & 520  & 93.1       & 0.001–0.26    & \citet{ismail2016seasonal}           \\
                & Pakistan     & Raw milk                & AFM1       & 960  & 70         & 0.3–1.0       & \citet{akbar2019occurrence}          \\
                & Pakistan     & Rice                    & AFB1       & 2047 & 73.3       & 1.17–6.91     & \citet{asghar2016incidence}          \\
                & Pakistan     & Rice \&  rice products  & AFB1       & 208  & 35.1       & 0.04–21.3     & \citet{iqbal2016presence}            \\
                &              &                         & \textSigma AFs &      & 35.1       & 0.04–32.2     &                                      \\
                & Pakistan     & Tea                     & \textSigma AFs & 94   & 78.3       & 0.11–16.17    & \citet{ismail2020prevalence}         \\ \hline
Southeast Asia  & Malaysia     & Cow milk                & AFM1       & 102  & 2          & 0.020–0.142   & \citet{shuib2017natural}             \\
                & Malaysia     & Milk \& milk products   & AFM1       & 53   & 35.8       & 0.0035–0.1005 & \citet{nadira2017screening}          \\
                & Malaysia     & Spices                  & \textSigma AFs & 34   & 85         & 0.01–9.34     & \citet{ali2015natural}               \\
                & Vietnam      & Maize                   & AFB1       & 2370 & 33.7       & 0.02–34.8     & \citet{lee2017survey}                \\ \hline
Southern Europe & Greece       & Milk                    & AFM 1      & 196  & 46.4       & 0.005–0.016   & \citet{tsakiris2013risk}             \\
                & Greece       & Sesame seeds            & AFB1       & 30   & 77.6       & 0.02–14.49    & \citet{kollia2016aflatoxin}          \\
                & Italy        & Cow milk                & AFM1       & 416  & 12.3       & 0.004–0.052   & \citet{de2017survey}                 \\
                &              & Buffalo milk            &            & 388  & 7.2        & 0.004–0.031   &                                      \\
                & Italy        & Spices                  & AFB1       & 130  & 15.4       & 0.59–5.38     & \citet{prelle2014co}                 \\
                & Kosovo       & Raw milk                & AFM1       & 826  & 2.8        & 0.005–0.05    & \citet{rama2016study}                \\
                &              & UHT milk                &            & 69   & 2.6        & 0.005–0.05    &                                      \\
                & Macedonia    & Raw milk                & AFM1       & 3635 & 42.4       & 0.0066–0.4081 & \citet{dimitrieska2016assessment}    \\
                & Portugal     & Milk                    & AFM1       & 40   & 27.5       & 0.005–0.069   & \citet{duarte2013aflatoxin}          \\
                & Serbia       & Corn                    & \textSigma AFs & 380  & 36.1       & 1.01–86.1     & \citet{kos2013natural}               \\
                & Serbia       & Milk                    & AFM 1      & 176  & 93.8       & 0.01–1.20     & \citet{kos2014occurrence}            \\
                & Serbia       & Maize                   & AFB1       & 56   & 48.2       & 0.04–8.80     & \citet{torovic2018aflatoxins}        \\
                &              &                         & \textSigma AFs &      & 48.2       & 0.04–9.14     &                                      \\
                & Serbia       & Milk                    & AFM1       & 80   & 92.5       & 0.003–0.319   & \citet{torovic2015aflatoxin}         \\
                &              & Infant formula          &            & 21   & 4.8        & 0.03–0.02     &                                      \\
                & Serbia       & Raw milk                & AFM1       & 678  & 100        & 0.025–>1      & \citet{tomavsevic2015two}            \\
                &              & Heat treated milk       &            & 438  & 100        & 0.025–1       &                                      \\
                &              & Milk products           &            & 322  & 100        & 0.025–>1      &                                      \\
                & Spain        & Cereals                 & \textSigma AFs & 67   & 0          & ND            & \citet{vidal2013determination}       \\
                & Spain        & Toasted cereal flour    & AFB1       & 94   & 25.5       & 0.025–0.17    & \citet{luzardo2016estimated}         \\
                &              &                         & AFB2       &      & 24.5       & 0.025–0.07    &                                      \\
                &              &                         & AFG1       &      & 9.6        & 0.025–0.12    &                                      \\
                &              &                         & AFG2       &      & 8.5        & 0.025–0.17    &                                      \\
                & Spain        & Wheat (pizza dough)     & AFB1       & 60   & 23         & 1.03–9.50     & \citet{quiles2016occurrence}         \\
                &              &                         & AFB2       &      & 32         & 0.34–0.67     &                                      \\
                & Turkey       & Cow milk                & AFM1       & 176  & 30.1       & 0.025–1.01    & \citet{golge2014survey}              \\
                & Turkey       & Hazelnuts               & \textSigma AFs & 170  & 6.5        & 0.09–11.3     & \citet{kabak2016aflatoxins}          \\
                &              & Dried figs              &            & 130  & 12.3       & 0.1–28.2      &                                      \\
                & Turkey       & Maize                   & \textSigma AFs & 1055 & 4          & 7.96–163.62   & \citet{artik2016aflatoxin}           \\
                & Turkey       & Peanut                  & \textSigma AFs & 102  & 84         & 0.2–2177.2    & \citet{lavkor2019presence}           \\
                & Turkey       & Wheat                   & \textSigma AFs & 141  & 2          & 0.21–0.44     & \citet{turksoy2020determination}     \\
                & Turkey       & Wheat flour             & AFB1       & 60   & 0          & ND            & \citet{kara2015co}                   \\
                &              & Maize flour             &            & 24   & 66.7       & 0.041–1.12    &                                      \\ \hline
Sub Saharan Africa &
  Burkina Faso &
  Sorghum malt &
  AFB1 &
  50 &
  25 &
  46.33–254.73 &
  \citet{bationo2015assessment} \\
                & Congo        & Corn (pre-harvest)      & \textSigma AFs & 50   & 32         & 3.1–103.89    & \citet{kamika2016occurrence}         \\
                &              & Corn (post harvest)     &            & 150  & 52         & 1.5–2806.5    &                                      \\
                & Ethiopia     & Groundnuts              & \textSigma AFs & 120  & 77.5       & 15–11900      & \citet{chala2013natural}             \\
                & Ethiopia     & Maize                   & \textSigma AFs & 150  & 100        & 20–91.04      & \citet{chauhan2016fungal}            \\
                & Ethiopia     & Peanut                  & AFB1       & 160  & 26.9       & 1.0–2526      & \citet{mohammed2016aspergillus}      \\
                &              &                         & AFB2       &      & 27.5       & 0.05–237      &                                      \\
                &              &                         & AFG1       &      & 5.3        & 1–736         &                                      \\
                &              &                         & AFG2       &      & 5.5        & 0.05–171      &                                      \\
                & Ethiopia     & Sorghum                 & AFB1       & 90   & 100        & 1–33.10       & \citet{taye2016aflatoxin}            \\
                & Ghana        & Maize                   & \textSigma AFs & 326  & 37.7       & 0.1–341       & \citet{agbetiameh2018prevalence}     \\
                & Kenya        & Pearl Millet            & AFB1       & 205  & 64         & 1.0–1658.2    & \citet{sirma2016aflatoxin}           \\
                & Kenya        & Raw milk                & AFM1       & 96   & 100        & 0.0154–4.563  & \citet{kuboka2019occurrence}         \\
                & Malawi       & Nut-based foods         & AFB1       & 55   & 78.2       & 0.1–40.6/6.28 & \citet{matumba2014survey}            \\
                & Namibia      & Sorghum malt            & AFB1       & 45   & 44         & 0.61–28.3     & \citet{nafuka2019variation}          \\
                &              &                         & AFB2       &      & 9          & 0.14–2.35     &                                      \\
                &              &                         & AFG1       &      & 17         & 0.39–6.95     &                                      \\
                & Nigeria      & Chilies                 & AFB1       & 55   & 38.2       & 2–156         & \citet{singh2017aflatoxin}           \\
                & Nigeria      & Ginger                  & AFB1       & 120  & 55         & 0.11–8.76     & \citet{lippolis2017natural}          \\
                &              &                         & AFB2       &      & 36.7       & 0.13–1.01     &                                      \\
                &              &                         & \textSigma AFs &      & 55         & 0.11–9.52     &                                      \\
                & Nigeria      & Peanut                  & AFB1       & 84   & 29.8       & 0.9–710       & \citet{oyedele2017mycotoxin}         \\
                &              &                         & AFB2       &      & 17.9       & 0.4–129       &                                      \\
                &              &                         & AFG1       &      & 22.6       & 0.4–1202      &                                      \\
                &              &                         & AFG2       &      & 7.1        & 18.3–123      &                                      \\
                &              &                         & \textSigma AFs &      & 39.3       & 0.4–2076      &                                      \\
                & Nigeria      & Rice                    & AFB1       & 38   & 18.4       & 0.15–20.2     & \citet{rofiat2015fungal}             \\
                &              &                         & AFB2       &      & 13.2       & 0.2–6.11      &                                      \\
                &              &                         & AFG1       &      & 5.3        & 0.2–7.21      &                                      \\
                & Nigeria      & Roasted cashew nuts     & \textSigma AFs & 27   & 100        & 0.1–6.8       & \citet{adetunji2018microbiological}  \\
                & Nigeria      & Sorghum                 & \textSigma AFs & 146  & 28.6       & 0.96–21.74    & \citet{daniel2016mycotoxicological}  \\
                & Togo         & Maize                   & AFB1       & 70   & 76         & 1.1–75.9      & \citet{hanvi2021aflatoxins}          \\
                & Uganda       & Maize                   & \textSigma AFs & 256  & 25.8       & 0–3760        & \citet{sserumaga2020aflatoxin}       \\
                & Zambia       & Peanuts                 & AFB1       & 92   & 44.6       & 0.015–46.60   & \citet{bumbangi2016occurrence}       \\
                &              &                         & \textSigma AFs &      & 55.4       & 0.014–48.67   &                                      \\
                & Zimbabwe     & Corn                    & AFB1       & 388  & 20.6       & 0.75–26.6     & \citet{murashiki2017levels}          \\ \hline             
\end{longtable}
\begin{minipage}{.8\linewidth}
%do not draw the footnoterule
\renewcommand{\footnoterule}
ND: Not detected \\
\textsuperscript{[a]} Aflatoxin concentrations are expressed in \textmu g kg \textsuperscript{-1} for solid matrices and \textmu g L \textsuperscript{-1} for liquid matrices.
\end{minipage}
\endgroup
\end{landscape}

 
%++++++++++++++++++++++++++++++++++++++++++++++++++++++++++++++++++++++++++++++++++++++++

%----------------------------------------------------------------------------------------
\subsection*{National and internationally harmonized limits for aflatoxins in foodstuffs} % Main appendix title
%----------------------------------------------------------------------------------------

In gain an overview of the changes in the legal limits for aflatoxins in food, both at national and international level, over a period of 20 years, a comparative analysis was carried out between the years 2002 and 2022. The data research revealed that in most countries, regulatory limits have been established for the sum of the four major aflatoxins (AFB1, AFB2, AFG1, and AFG2) and/or for the most toxic aflatoxin (AFB1), particularly in the context of maize and peanuts. Consequently, data were collected for these specific foods and parameters. In 2002, the Food and Agriculture Organization of the United Nations (FAO) made an important contribution in this area by conducting a comprehensive study aimed at assessing the global landscape of mycotoxin regulation \citep{van2004worldwide}. This study found that for corn and peanuts, a total of 89 countries have set limits. Of these, 67 countries set national standards, while 22 countries adopted internationally harmonized standards within economic unions such as the European Union (EU), Mercosur (Common Market of the South) and Australia/New Zealand. In the following years, more countries set their own national standards, and more nations joined these economic unions, e.g. through the eastward expansion of the EU. In addition, the emergence of economic unions such as ARSO (African Organization for Standardization), EAC (East African Community), EACU (Eurasian Customs Union) and GSO (Gulf Cooperation Council Standardization Organization) contributed to the implementation of limits by more countries. The following table shows the status of regulation at national and international level for different countries and time periods. This data was then used for creating world  World maps (\ref{fig:Aflatoxin_Regulation_Limits}). In cases where multiple limits existed for a given time period and nation, the most stringent value was used for visualization.

\clearpage

%++++++++++++++++++++++++++++++++++++++++++++++++++++++++++++++++++++++++++++++++++++++++
\begin{landscape}
\begingroup\footnotesize
\begin{longtable}[c]{llllllll}
\captionsetup{labelfont=bf, justification=justified, singlelinecheck=false, width=1.5\textwidth} 
\caption{The situation of aflatoxin regulation in the year 2002 and 2022: Regulation limits for AFB1 and the sum of AFB1, AFB2, AFG1, and AFG2 in maize and peanuts, considering both countries that follow internationally harmonized aflatoxin standards and those that set their own national limits.} 
\\
\label{table:Aflatoxin_regulation}
\\
\hline
\textbf{Country} &
  \textbf{Hierarchy} &
  \textbf{Economic union} &
  \textbf{Time stamp} &
  \textbf{Food} &
  \textbf{AFB1\textsuperscript{[a]}} &
  \textbf{\textSigma AFs\textsuperscript{[a]}} &
  \textbf{Reference} \\ \hline
\endfirsthead
%
\multicolumn{8}{c}%
{{\bfseries Table \thetable\ continued from previous page}} \\
\hline
\textbf{Country} &
  \textbf{Hierarchy} &
  \textbf{Economic union} &
  \textbf{Time stamp} &
  \textbf{Food} &
  \textbf{AFB1} &
  \textbf{sumAFs} &
  \textbf{Reference} \\ \hline
\endhead
%
\hline
\endfoot
%
\endlastfoot
%
Algeria           & National      & -        & 2002 & Maize  & 10 & 20 & \citet{van2004worldwide}     \\
Algeria           & International & ARSO     & 2022 & Maize  & 5  & 10 & \citet{ARSO2022}             \\
Algeria           & National      & -        & 2002 & Peanut & 10 & 20 & \citet{van2004worldwide}     \\
Algeria           & International & ARSO     & 2022 & Peanut & 5  & 10 & \citet{ARSO2022}             \\
Argentina         & International & Mercosur & 2002 & Maize  & NA & 20 & \citet{MERCOSUR2002}         \\
Argentina         & International & Mercosur & 2002 & Peanut & NA & 20 & \citet{MERCOSUR2002}         \\
Armenia           & International & EACU     & 2022 & Maize  & 5  & NA & \citet{EACU2011}             \\
Armenia           & National      & -        & 2002 & Peanut & 5  & NA & \citet{van2004worldwide}     \\
Armenia           & International & EACU     & 2022 & Peanut & 5  & NA & \citet{EACU2011}             \\
Armenia           & National      & -        & 2002 & Maize  & 5  & NA & \citet{van2004worldwide}     \\
Australia         & International & AU\&NZ    & 2002 & Peanut & NA & 15 & \citet{FSANZ2022}            \\
Austria           & International & EU       & 2002 & Maize  & 2  & 4  & \citet{EC2010}               \\
Austria           & International & EU       & 2022 & Maize  & 2  & 4  & \citet{EC2010}               \\
Austria           & International & EU       & 2002 & Peanut & 2  & 4  & \citet{EC2010}               \\
Austria           & International & EU       & 2022 & Peanut & 2  & 4  & \citet{EC2010}               \\
Bahrain           & International & -        & 2022 & Maize  & NA & 4  & \citet{van2004worldwide}     \\
Bahrain           & International & -        & 2022 & Peanut & NA & 4  & \citet{van2004worldwide}     \\
Bangladesh        & National      & -        & 2022 & Peanut & NA & 10 & \citet{BFSA2017}             \\
Barbados          & National      & -        & 2002 & Maize  & NA & 20 & \citet{van2004worldwide}     \\
Barbados          & National      & -        & 2002 & Peanut & NA & 20 & \citet{van2004worldwide}     \\
Belarus           & International & EACU     & 2022 & Maize  & 5  & NA & \citet{EACU2011}             \\
Belarus           & International & EACU     & 2022 & Peanut & 5  & NA & \citet{EACU2011}             \\
Belgium           & International & EU       & 2002 & Maize  & 2  & 4  & \citet{EC2010}               \\
Belgium           & International & EU       & 2022 & Maize  & 2  & 4  & \citet{EC2010}               \\
Belgium           & International & EU       & 2002 & Peanut & 2  & 4  & \citet{EC2010}               \\
Belgium           & International & EU       & 2022 & Peanut & 2  & 4  & \citet{EC2010}               \\
Belize            & National      & -        & 2002 & Maize  & NA & 20 & \citet{van2004worldwide}     \\
Belize            & National      & -        & 2002 & Peanut & NA & 20 & \citet{van2004worldwide}     \\
Benin             & International & ARSO     & 2022 & Maize  & 5  & 10 & \citet{ARSO2022}             \\
Benin             & International & ARSO     & 2022 & Peanut & 5  & 10 & \citet{ARSO2022}             \\
Botswana          & International & ARSO     & 2022 & Maize  & 5  & 10 & \citet{ARSO2022}             \\
Botswana          & International & ARSO     & 2022 & Peanut & 5  & 10 & \citet{ARSO2022}             \\
Brazil            & International & Mercosur & 2002 & Maize  & NA & 20 & \citet{MERCOSUR2002}         \\
Brazil            & International & Mercosur & 2002 & Peanut & NA & 20 & \citet{MERCOSUR2002}         \\
Bulgaria          & National      & -        & 2002 & Maize  & 2  & 4  & \citet{van2004worldwide}     \\
Bulgaria          & International & EU       & 2022 & Maize  & 2  & 4  & \citet{EC2010}               \\
Bulgaria          & National      & -        & 2002 & Peanut & 2  & 4  & \citet{van2004worldwide}     \\
Bulgaria          & International & EU       & 2022 & Peanut & 2  & 4  & \citet{EC2010}               \\
Burkina Faso      & International & ARSO     & 2022 & Maize  & 5  & 10 & \citet{ARSO2022}             \\
Burkina Faso      & International & ARSO     & 2022 & Peanut & 5  & 10 & \citet{ARSO2022}             \\
Burundi           & International & ARSO     & 2022 & Maize  & 5  & 10 & \citet{ARSO2022}             \\
Burundi           & International & EAC      & 2022 & Maize  & 5  & 10 & \citet{EAC2018}              \\
Burundi           & International & ARSO     & 2022 & Peanut & 5  & 10 & \citet{ARSO2022}             \\
Burundi           & International & EAC      & 2022 & Peanut & 5  & 10 & \citet{EAC2018}              \\
Cameroon          & International & ARSO     & 2022 & Maize  & 5  & 10 & \citet{ARSO2022}             \\
Cameroon          & International & ARSO     & 2022 & Peanut & 5  & 10 & \citet{ARSO2022}             \\
Canada            & National      & -        & 2002 & Peanut & NA & 15 & \citet{van2004worldwide}     \\
Chad              & International & ARSO     & 2022 & Maize  & 5  & 10 & \citet{ARSO2022}             \\
Chad              & International & ARSO     & 2022 & Peanut & 5  & 10 & \citet{ARSO2022}             \\
Chile             & National      & -        & 2002 & Maize  & NA & 5  & \citet{van2004worldwide}     \\
Chile             & National      & -        & 2002 & Peanut & NA & 5  & \citet{van2004worldwide}     \\
China             & National      & -        & 2002 & Maize  & 20 & NA & \citet{van2004worldwide}     \\
Colombia          & National      & -        & 2002 & Maize  & NA & 20 & \citet{van2004worldwide}     \\
Colombia          & National      & -        & 2002 & Peanut & NA & 10 & \citet{van2004worldwide}     \\
Costa Rica        & National      & -        & 2002 & Maize  & NA & 35 & \citet{van2004worldwide}     \\
Costa Rica        & National      & -        & 2022 & Maize  & NA & 20 & \citet{MHCR2011a}            \\
Costa Rica        & National      & -        & 2022 & Peanut & NA & 15 & \citet{MHCR2011b}            \\
Croatia           & International & EU       & 2002 & Maize  & 2  & 4  & \citet{EC2010}               \\
Croatia           & National      & -        & 2002 & Maize  & 5  & NA & \citet{van2004worldwide}     \\
Croatia           & International & EU       & 2022 & Maize  & 2  & 4  & \citet{EC2010}               \\
Croatia           & International & EU       & 2002 & Peanut & 2  & 4  & \citet{EC2010}               \\
Croatia           & National      & -        & 2002 & Peanut & 5  & NA & \citet{van2004worldwide}     \\
Croatia           & International & EU       & 2022 & Peanut & 2  & 4  & \citet{EC2010}               \\
Cuba              & National      & -        & 2002 & Maize  & 5  & 5  & \citet{van2004worldwide}     \\
Cuba              & National      & -        & 2002 & Peanut & 5  & 5  & \citet{van2004worldwide}     \\
Cyprus            & National      & -        & 2002 & Maize  & 2  & 4  & \citet{van2004worldwide}     \\
Cyprus            & International & EU       & 2022 & Maize  & 2  & 4  & \citet{EC2010}               \\
Cyprus            & National      & -        & 2002 & Peanut & 2  & 4  & \citet{van2004worldwide}     \\
Cyprus            & International & EU       & 2022 & Peanut & 2  & 4  & \citet{EC2010}               \\
Czech Republic    & International & EU       & 2022 & Maize  & 2  & 4  & \citet{EC2010}               \\
Czech Republic    & International & EU       & 2022 & Peanut & 2  & 4  & \citet{EC2010}               \\
CzechRepublic     & National      & -        & 2002 & Maize  & 2  & 4  & \citet{van2004worldwide}     \\
CzechRepublic     & National      & -        & 2002 & Peanut & 2  & 4  & \citet{van2004worldwide}     \\
Democratic Republic of the Congo &
  International &
  ARSO &
  2022 &
  Maize &
  5 &
  10 &
  \citet{ARSO2022} \\
Democratic Republic of the Congo &
  International &
  EAC &
  2022 &
  Maize &
  5 &
  10 &
  \citet{EAC2018} \\
Democratic Republic of the Congo &
  International &
  ARSO &
  2022 &
  Peanut &
  5 &
  10 &
  \citet{ARSO2022} \\
Democratic Republic of the Congo &
  International &
  EAC &
  2022 &
  Peanut &
  5 &
  10 &
  \citet{EAC2018} \\
Denmark           & International & EU       & 2002 & Maize  & 2  & 4  & \citet{EC2010}               \\
Denmark           & International & EU       & 2022 & Maize  & 2  & 4  & \citet{EC2010}               \\
Denmark           & International & EU       & 2002 & Peanut & 2  & 4  & \citet{EC2010}               \\
Denmark           & International & EU       & 2022 & Peanut & 2  & 4  & \citet{EC2010}               \\
Djibouti          & International & ARSO     & 2022 & Maize  & 5  & 10 & \citet{ARSO2022}             \\
Djibouti          & International & ARSO     & 2022 & Peanut & 5  & 10 & \citet{ARSO2022}             \\
Ecuador           & National      & -        & 2002 & Maize  & 10 & 20 & \citet{van2004worldwide}     \\
Ecuador           & National      & -        & 2002 & Peanut & 5  & 10 & \citet{van2004worldwide}     \\
Egypt             & International & ARSO     & 2022 & Maize  & 5  & 10 & \citet{ARSO2022}             \\
Egypt             & International & ARSO     & 2022 & Peanut & 5  & 10 & \citet{ARSO2022}             \\
Estonia           & National      & -        & 2002 & Maize  & 2  & 4  & \citet{van2004worldwide}     \\
Estonia           & International & EU       & 2022 & Maize  & 2  & 4  & \citet{EC2010}               \\
Estonia           & National      & -        & 2002 & Peanut & 2  & 4  & \citet{van2004worldwide}     \\
Estonia           & International & EU       & 2022 & Peanut & 2  & 4  & \citet{EC2010}               \\
Ethiopia          & International & ARSO     & 2022 & Maize  & 5  & 10 & \citet{ARSO2022}             \\
Ethiopia          & International & ARSO     & 2022 & Peanut & 5  & 10 & \citet{ARSO2022}             \\
Finland           & International & EU       & 2002 & Maize  & 2  & 4  & \citet{EC2010}               \\
Finland           & International & EU       & 2022 & Maize  & 2  & 4  & \citet{EC2010}               \\
Finland           & International & EU       & 2002 & Peanut & 2  & 4  & \citet{EC2010}               \\
Finland           & International & EU       & 2022 & Peanut & 2  & 4  & \citet{EC2010}               \\
France            & International & EU       & 2002 & Maize  & 2  & 4  & \citet{EC2010}               \\
France            & International & EU       & 2022 & Maize  & 2  & 4  & \citet{EC2010}               \\
France            & International & EU       & 2002 & Peanut & 2  & 4  & \citet{EC2010}               \\
France            & International & EU       & 2022 & Peanut & 2  & 4  & \citet{EC2010}               \\
Gabon             & International & ARSO     & 2022 & Maize  & 5  & 10 & \citet{ARSO2022}             \\
Gabon             & International & ARSO     & 2022 & Peanut & 5  & 10 & \citet{ARSO2022}             \\
Germany           & International & EU       & 2002 & Maize  & 2  & 4  & \citet{EC2010}               \\
Germany           & International & EU       & 2022 & Maize  & 2  & 4  & \citet{EC2010}               \\
Germany           & International & EU       & 2002 & Peanut & 2  & 4  & \citet{EC2010}               \\
Germany           & International & EU       & 2022 & Peanut & 2  & 4  & \citet{EC2010}               \\
Ghana             & International & ARSO     & 2022 & Maize  & 5  & 10 & \citet{ARSO2022}             \\
Ghana             & International & ARSO     & 2022 & Peanut & 5  & 10 & \citet{ARSO2022}             \\
Greece            & International & EU       & 2002 & Maize  & 2  & 4  & \citet{EC2010}               \\
Greece            & International & EU       & 2022 & Maize  & 2  & 4  & \citet{EC2010}               \\
Greece            & International & EU       & 2002 & Peanut & 2  & 4  & \citet{EC2010}               \\
Greece            & International & EU       & 2022 & Peanut & 2  & 4  & \citet{EC2010}               \\
Guatemala         & National      & -        & 2002 & Maize  & NA & 20 & \citet{van2004worldwide}     \\
Guatemala         & National      & -        & 2002 & Peanut & NA & 20 & \citet{van2004worldwide}     \\
Guinea            & International & ARSO     & 2022 & Maize  & 5  & 10 & \citet{ARSO2022}             \\
Guinea            & International & ARSO     & 2022 & Peanut & 5  & 10 & \citet{ARSO2022}             \\
Guinea-Bissau     & International & ARSO     & 2022 & Maize  & 5  & 10 & \citet{ARSO2022}             \\
Guinea-Bissau     & International & ARSO     & 2022 & Peanut & 5  & 10 & \citet{ARSO2022}             \\
Honduras          & National      & -        & 2002 & Maize  & 1  & 1  & \citet{van2004worldwide}     \\
Honduras          & National      & -        & 2002 & Peanut & NA & 1  & \citet{van2004worldwide}     \\
Hong Kong         & National      & -        & 2002 & Maize  & 15 & 15 & \citet{van2004worldwide}     \\
Hong Kong         & National      & -        & 2002 & Peanut & 20 & 20 & \citet{van2004worldwide}     \\
Hungary           & National      & -        & 2002 & Maize  & 2  & 4  & \citet{van2004worldwide}     \\
Hungary           & National      & -        & 2002 & Maize  & 2  & 4  & \citet{van2004worldwide}     \\
Hungary           & International & EU       & 2022 & Maize  & 2  & 4  & \citet{EC2010}               \\
Hungary           & National      & -        & 2002 & Peanut & 2  & 4  & \citet{van2004worldwide}     \\
Hungary           & International & EU       & 2022 & Peanut & 2  & 4  & \citet{EC2010}               \\
India             & National      & -        & 2002 & Maize  & NA & 30 & \citet{van2004worldwide}     \\
India             & National      & -        & 2022 & Maize  & 10 & 15 & \citet{FSSAI2011}            \\
India             & National      & -        & 2002 & Peanut & NA & 30 & \citet{van2004worldwide}     \\
India             & National      & -        & 2022 & Peanut & 10 & 15 & \citet{FSSAI2011}            \\
Indonesia         & National      & -        & 2022 & Maize  & 15 & 20 & \citet{MOA2018}              \\
Indonesia         & National      & -        & 2002 & Peanut & NA & 20 & \citet{van2004worldwide}     \\
Indonesia         & National      & -        & 2022 & Peanut & 15 & 20 & \citet{MOA2018}              \\
Iran              & National      & -        & 2002 & Maize  & 5  & 30 & \citet{van2004worldwide}     \\
Iran              & National      & -        & 2022 & Maize  & 5  & 30 & \citet{ISIRI2002}            \\
Iran              & National      & -        & 2002 & Peanut & 5  & 15 & \citet{van2004worldwide}     \\
Iran              & National      & -        & 2022 & Peanut & 5  & 15 & \citet{ISIRI2002}            \\
Ireland           & International & EU       & 2002 & Maize  & 2  & 4  & \citet{EC2010}               \\
Ireland           & International & EU       & 2022 & Maize  & 2  & 4  & \citet{EC2010}               \\
Ireland           & International & EU       & 2002 & Peanut & 2  & 4  & \citet{EC2010}               \\
Ireland           & International & EU       & 2022 & Peanut & 2  & 4  & \citet{EC2010}               \\
Israel            & National      & -        & 2002 & Maize  & 5  & 15 & \citet{van2004worldwide}     \\
Israel            & National      & -        & 2002 & Peanut & 5  & 15 & \citet{van2004worldwide}     \\
Italy             & International & EU       & 2002 & Maize  & 2  & 4  & \citet{EC2010}               \\
Italy             & International & EU       & 2022 & Maize  & 2  & 4  & \citet{EC2010}               \\
Italy             & International & EU       & 2002 & Peanut & 2  & 4  & \citet{EC2010}               \\
Italy             & International & EU       & 2022 & Peanut & 2  & 4  & \citet{EC2010}               \\
Ivory Coast       & International & ARSO     & 2022 & Maize  & 5  & 10 & \citet{ARSO2022}             \\
Ivory Coast       & International & ARSO     & 2022 & Peanut & 5  & 10 & \citet{ARSO2022}             \\
Jamaica           & National      & -        & 2002 & Maize  & NA & 20 & \citet{van2004worldwide}     \\
Jamaica           & National      & -        & 2002 & Peanut & NA & 20 & \citet{van2004worldwide}     \\
Japan             & National      & -        & 2002 & Maize  & NA & 10 & \citet{van2004worldwide}     \\
Japan             & National      & -        & 2022 & Maize  & 10 & NA & \citet{FAMIC2015}            \\
Japan             & National      & -        & 2002 & Peanut & NA & 10 & \citet{van2004worldwide}     \\
Japan             & National      & -        & 2022 & Peanut & 10 & NA & \citet{FAMIC2015}            \\
Jordan            & National      & -        & 2002 & Maize  & 15 & 30 & \citet{van2004worldwide}     \\
Jordan            & National      & -        & 2002 & Peanut & 15 & 30 & \citet{van2004worldwide}     \\
Kazakhstan        & International & EACU     & 2022 & Maize  & 5  & NA & \citet{EACU2011}             \\
Kazakhstan        & International & EACU     & 2022 & Peanut & 5  & NA & \citet{EACU2011}             \\
Kenya             & International & ARSO     & 2022 & Maize  & 5  & 10 & \citet{ARSO2022}             \\
Kenya             & International & EAC      & 2022 & Maize  & 5  & 10 & \citet{EAC2018}              \\
Kenya             & National      & -        & 2002 & Peanut & NA & 20 & \citet{van2004worldwide}     \\
Kenya             & International & ARSO     & 2022 & Peanut & 5  & 10 & \citet{ARSO2022}             \\
Kenya             & International & EAC      & 2022 & Peanut & 5  & 10 & \citet{EAC2018}              \\
Kuweit            & International & -        & 2022 & Maize  & NA & 4  & \citet{van2004worldwide}     \\
Kuweit            & International & -        & 2022 & Peanut & NA & 4  & \citet{van2004worldwide}     \\
Kyrgyzstan        & International & EACU     & 2022 & Maize  & 5  & NA & \citet{EACU2011}             \\
Kyrgyzstan        & International & EACU     & 2022 & Peanut & 5  & NA & \citet{EACU2011}             \\
Latvia            & National      & -        & 2002 & Maize  & 2  & 4  & \citet{van2004worldwide}     \\
Latvia            & National      & -        & 2002 & Maize  & 5  & NA & \citet{van2004worldwide}     \\
Latvia            & International & EU       & 2022 & Maize  & 2  & 4  & \citet{EC2010}               \\
Latvia            & National      & -        & 2002 & Peanut & 2  & 4  & \citet{van2004worldwide}     \\
Latvia            & National      & -        & 2002 & Peanut & 5  & NA & \citet{van2004worldwide}     \\
Latvia            & International & EU       & 2022 & Peanut & 2  & 4  & \citet{EC2010}               \\
Liberia           & International & ARSO     & 2022 & Maize  & 5  & 10 & \citet{ARSO2022}             \\
Liberia           & International & ARSO     & 2022 & Peanut & 5  & 10 & \citet{ARSO2022}             \\
Libya             & International & ARSO     & 2022 & Maize  & 5  & 10 & \citet{ARSO2022}             \\
Libya             & International & ARSO     & 2022 & Peanut & 5  & 10 & \citet{ARSO2022}             \\
Lithuania         & National      & -        & 2002 & Maize  & 2  & 4  & \citet{van2004worldwide}     \\
Lithuania         & National      & -        & 2002 & Maize  & 2  & 4  & \citet{van2004worldwide}     \\
Lithuania         & International & EU       & 2022 & Maize  & 2  & 4  & \citet{EC2010}               \\
Lithuania         & National      & -        & 2002 & Peanut & 2  & 4  & \citet{van2004worldwide}     \\
Lithuania         & National      & -        & 2002 & Peanut & 2  & 4  & \citet{van2004worldwide}     \\
Lithuania         & International & EU       & 2022 & Peanut & 2  & 4  & \citet{EC2010}               \\
Luxembourg        & International & EU       & 2002 & Maize  & 2  & 4  & \citet{EC2010}               \\
Luxembourg        & International & EU       & 2022 & Maize  & 2  & 4  & \citet{EC2010}               \\
Luxembourg        & International & EU       & 2002 & Peanut & 2  & 4  & \citet{EC2010}               \\
Luxembourg        & International & EU       & 2022 & Peanut & 2  & 4  & \citet{EC2010}               \\
Madagascar        & International & ARSO     & 2022 & Maize  & 5  & 10 & \citet{ARSO2022}             \\
Madagascar        & International & ARSO     & 2022 & Peanut & 5  & 10 & \citet{ARSO2022}             \\
Malawi            & International & ARSO     & 2022 & Maize  & 5  & 10 & \citet{ARSO2022}             \\
Malawi            & National      & -        & 2002 & Peanut & 5  & NA & \citet{van2004worldwide}     \\
Malawi            & International & ARSO     & 2022 & Peanut & 5  & 10 & \citet{ARSO2022}             \\
Malawi            & National      & -        & 2022 & Peanut & NA & 3  & \citet{chilaka2022mycotoxin} \\
Malaysia          & National      & -        & 2002 & Maize  & NA & 35 & \citet{van2004worldwide}     \\
Malaysia          & National      & -        & 2022 & Maize  & NA & 5  & \citet{MOH2014}              \\
Malaysia          & National      & -        & 2002 & Peanut & NA & 35 & \citet{van2004worldwide}     \\
Malaysia          & National      & -        & 2022 & Peanut & NA & 10 & \citet{MOH2014}              \\
Malta             & National      & -        & 2002 & Maize  & 2  & 4  & \citet{van2004worldwide}     \\
Malta             & National      & -        & 2002 & Maize  & 2  & 4  & \citet{van2004worldwide}     \\
Malta             & International & EU       & 2022 & Maize  & 2  & 4  & \citet{EC2010}               \\
Malta             & National      & -        & 2002 & Peanut & 2  & 4  & \citet{van2004worldwide}     \\
Malta             & National      & -        & 2002 & Peanut & 2  & 4  & \citet{van2004worldwide}     \\
Malta             & International & EU       & 2022 & Peanut & 2  & 4  & \citet{EC2010}               \\
Mauritius         & National      & -        & 2002 & Maize  & 5  & 10 & \citet{van2004worldwide}     \\
Mauritius         & International & ARSO     & 2022 & Maize  & 5  & 10 & \citet{ARSO2022}             \\
Mauritius         & National      & -        & 2002 & Peanut & 5  & 15 & \citet{van2004worldwide}     \\
Mauritius         & International & ARSO     & 2022 & Peanut & 5  & 10 & \citet{ARSO2022}             \\
Mexico            & National      & -        & 2002 & Maize  & NA & 20 & \citet{van2004worldwide}     \\
Mexico            & National      & -        & 2002 & Peanut & NA & 20 & \citet{van2004worldwide}     \\
Moldova           & National      & -        & 2002 & Maize  & 5  & NA & \citet{van2004worldwide}     \\
Moldova           & National      & -        & 2002 & Peanut & 5  & NA & \citet{van2004worldwide}     \\
Morocco           & National      & -        & 2002 & Maize  & 5  & NA & \citet{van2004worldwide}     \\
Morocco           & International & ARSO     & 2022 & Maize  & 5  & 10 & \citet{ARSO2022}             \\
Morocco           & National      & -        & 2022 & Maize  & 2  & 4  & \citet{chilaka2022mycotoxin} \\
Morocco           & National      & -        & 2002 & Peanut & 1  & NA & \citet{van2004worldwide}     \\
Morocco           & International & ARSO     & 2022 & Peanut & 5  & 10 & \citet{ARSO2022}             \\
Morocco           & National      & -        & 2022 & Peanut & 2  & 4  & \citet{chilaka2022mycotoxin} \\
Mozambique        & National      & -        & 2002 & Peanut & NA & 10 & \citet{van2004worldwide}     \\
Namibia           & International & ARSO     & 2022 & Maize  & 5  & 10 & \citet{ARSO2022}             \\
Namibia           & International & ARSO     & 2022 & Peanut & 5  & 10 & \citet{ARSO2022}             \\
Nepal             & National      & -        & 2002 & Maize  & 20 & NA & \citet{van2004worldwide}     \\
Netherlands       & International & EU       & 2002 & Maize  & 2  & 4  & \citet{EC2010}               \\
Netherlands       & International & EU       & 2022 & Maize  & 2  & 4  & \citet{EC2010}               \\
Netherlands       & International & EU       & 2002 & Peanut & 2  & 4  & \citet{EC2010}               \\
Netherlands       & International & EU       & 2022 & Peanut & 2  & 4  & \citet{EC2010}               \\
New Zealand       & International & AU\&NZ    & 2002 & Peanut & NA & 15 & \citet{FSANZ2022}            \\
Niger             & International & ARSO     & 2022 & Maize  & 5  & 10 & \citet{ARSO2022}             \\
Niger             & International & ARSO     & 2022 & Peanut & 5  & 10 & \citet{ARSO2022}             \\
Nigeria           & National      & -        & 2002 & Maize  & 20 & NA & \citet{van2004worldwide}     \\
Nigeria           & International & ARSO     & 2022 & Maize  & 5  & 10 & \citet{ARSO2022}             \\
Nigeria           & National      & -        & 2002 & Peanut & 20 & NA & \citet{van2004worldwide}     \\
Nigeria           & International & ARSO     & 2022 & Peanut & 5  & 10 & \citet{ARSO2022}             \\
North Macedonia   & National      & -        & 2022 & Maize  & 2  & 4  & \citet{AHV2013}              \\
North Macedonia   & National      & -        & 2022 & Peanut & 2  & 4  & \citet{AHV2013}              \\
Norway            & National      & -        & 2002 & Maize  & 2  & 4  & \citet{van2004worldwide}     \\
Norway            & National      & -        & 2002 & Peanut & 2  & 4  & \citet{van2004worldwide}     \\
Oman              & National      & -        & 2002 & Maize  & 10 & NA & \citet{van2004worldwide}     \\
Oman              & International & -        & 2022 & Maize  & NA & 4  & \citet{van2004worldwide}     \\
Oman              & National      & -        & 2002 & Peanut & 10 & NA & \citet{van2004worldwide}     \\
Oman              & International & -        & 2022 & Peanut & NA & 4  & \citet{van2004worldwide}     \\
Paraguay          & International & Mercosur & 2002 & Maize  & NA & 20 & \citet{MERCOSUR2002}         \\
Paraguay          & International & Mercosur & 2002 & Peanut & NA & 20 & \citet{MERCOSUR2002}         \\
Peru              & National      & -        & 2002 & Peanut & NA & 15 & \citet{van2004worldwide}     \\
Philippines       & National      & -        & 2022 & Maize  & NA & 20 & \citet{BAFPS2015}            \\
Philippines       & National      & -        & 2002 & Peanut & NA & 20 & \citet{van2004worldwide}     \\
Philippines       & National      & -        & 2022 & Peanut & NA & 15 & \citet{BAFPS2015}            \\
Poland            & National      & -        & 2002 & Maize  & 2  & 4  & \citet{van2004worldwide}     \\
Poland            & International & EU       & 2022 & Maize  & 2  & 4  & \citet{EC2010}               \\
Poland            & National      & -        & 2002 & Peanut & 2  & 4  & \citet{van2004worldwide}     \\
Poland            & International & EU       & 2022 & Peanut & 2  & 4  & \citet{EC2010}               \\
Portugal          & International & EU       & 2002 & Maize  & 2  & 4  & \citet{EC2010}               \\
Portugal          & International & EU       & 2022 & Maize  & 2  & 4  & \citet{EC2010}               \\
Portugal          & International & EU       & 2002 & Peanut & 2  & 4  & \citet{EC2010}               \\
Portugal          & International & EU       & 2022 & Peanut & 2  & 4  & \citet{EC2010}               \\
Qatar             & International & -        & 2022 & Maize  & NA & 4  & \citet{van2004worldwide}     \\
Qatar             & International & -        & 2022 & Peanut & NA & 4  & \citet{van2004worldwide}     \\
Republic of Congo & International & ARSO     & 2022 & Maize  & 5  & 10 & \citet{ARSO2022}             \\
Republic of Congo & International & ARSO     & 2022 & Peanut & 5  & 10 & \citet{ARSO2022}             \\
Romania           & National      & -        & 2002 & Maize  & 5  & NA & \citet{van2004worldwide}     \\
Romania           & International & EU       & 2022 & Maize  & 2  & 4  & \citet{EC2010}               \\
Romania           & National      & -        & 2002 & Peanut & 5  & NA & \citet{van2004worldwide}     \\
Romania           & International & EU       & 2022 & Peanut & 2  & 4  & \citet{EC2010}               \\
Russia            & National      & -        & 2002 & Maize  & 5  & NA & \citet{van2004worldwide}     \\
Russia            & International & EACU     & 2022 & Maize  & 5  & NA & \citet{EACU2011}             \\
Russia            & National      & -        & 2002 & Peanut & 5  & NA & \citet{van2004worldwide}     \\
Russia            & International & EACU     & 2022 & Peanut & 5  & NA & \citet{EACU2011}             \\
Rwanda            & International & ARSO     & 2022 & Maize  & 5  & 10 & \citet{ARSO2022}             \\
Rwanda            & International & EAC      & 2022 & Maize  & 5  & 10 & \citet{EAC2018}              \\
Rwanda            & International & ARSO     & 2022 & Peanut & 5  & 10 & \citet{ARSO2022}             \\
Rwanda            & International & EAC      & 2022 & Peanut & 5  & 10 & \citet{EAC2018}              \\
Salvador          & National      & -        & 2002 & Maize  & NA & 20 & \citet{van2004worldwide}     \\
Salvador          & National      & -        & 2002 & Peanut & NA & 20 & \citet{van2004worldwide}     \\
Saudi Arabia      & International & -        & 2022 & Maize  & NA & 4  & \citet{van2004worldwide}     \\
Saudi Arabia      & International & -        & 2022 & Peanut & NA & 4  & \citet{van2004worldwide}     \\
Senegal           & International & ARSO     & 2022 & Maize  & 5  & 10 & \citet{ARSO2022}             \\
Senegal           & International & ARSO     & 2022 & Peanut & 5  & 10 & \citet{ARSO2022}             \\
Serbia            & National      & -        & 2002 & Maize  & 5  & NA & \citet{van2004worldwide}     \\
Serbia            & National      & -        & 2022 & Maize  & 2  & 4  & \citet{RS2019}               \\
Serbia            & National      & -        & 2002 & Peanut & 5  & NA & \citet{van2004worldwide}     \\
Serbia            & National      & -        & 2022 & Peanut & 2  & 4  & \citet{RS2019}               \\
Seychelles        & International & ARSO     & 2022 & Maize  & 5  & 10 & \citet{ARSO2022}             \\
Seychelles        & International & ARSO     & 2022 & Peanut & 5  & 10 & \citet{ARSO2022}             \\
Sierra Leone      & International & ARSO     & 2022 & Maize  & 5  & 10 & \citet{ARSO2022}             \\
Sierra Leone      & International & ARSO     & 2022 & Peanut & 5  & 10 & \citet{ARSO2022}             \\
Singapore         & National      & -        & 2002 & Maize  & NA & 5  & \citet{van2004worldwide}     \\
Singapore         & National      & -        & 2022 & Maize  & 5  & 5  & \citet{SFA2019}              \\
Singapore         & National      & -        & 2002 & Peanut & NA & 5  & \citet{van2004worldwide}     \\
Singapore         & National      & -        & 2022 & Peanut & 5  & 5  & \citet{SFA2019}              \\
Slovakia          & National      & -        & 2002 & Maize  & 20 & 80 & \citet{van2004worldwide}     \\
Slovakia          & International & EU       & 2022 & Maize  & 2  & 4  & \citet{EC2010}               \\
Slovakia          & National      & -        & 2002 & Peanut & 10 & 80 & \citet{van2004worldwide}     \\
Slovakia          & International & EU       & 2022 & Peanut & 2  & 4  & \citet{EC2010}               \\
Slovenia          & National      & -        & 2002 & Maize  & 2  & 4  & \citet{van2004worldwide}     \\
Slovenia          & International & EU       & 2022 & Maize  & 2  & 4  & \citet{EC2010}               \\
Slovenia          & National      & -        & 2002 & Peanut & 2  & 4  & \citet{van2004worldwide}     \\
Slovenia          & International & EU       & 2022 & Peanut & 2  & 4  & \citet{EC2010}               \\
Somalia           & International & ARSO     & 2022 & Maize  & 5  & 10 & \citet{ARSO2022}             \\
Somalia           & International & ARSO     & 2022 & Peanut & 5  & 10 & \citet{ARSO2022}             \\
South Africa      & National      & -        & 2002 & Maize  & 5  & 10 & \citet{van2004worldwide}     \\
South Africa      & International & ARSO     & 2022 & Maize  & 5  & 10 & \citet{ARSO2022}             \\
South Africa      & National      & -        & 2002 & Peanut & 5  & 10 & \citet{van2004worldwide}     \\
South Africa      & International & ARSO     & 2022 & Peanut & 5  & 10 & \citet{ARSO2022}             \\
South Korea       & National      & -        & 2002 & Maize  & 10 & NA & \citet{van2004worldwide}     \\
South Korea       & National      & -        & 2022 & Maize  & 10 & 15 & \citet{MFDS2019}             \\
South Korea       & National      & -        & 2002 & Peanut & 10 & NA & \citet{van2004worldwide}     \\
South Korea       & National      & -        & 2022 & Peanut & 10 & 15 & \citet{MFDS2019}             \\
South Sudan       & International & ARSO     & 2022 & Maize  & 5  & 10 & \citet{ARSO2022}             \\
South Sudan       & International & EAC      & 2022 & Maize  & 5  & 10 & \citet{EAC2018}              \\
South Sudan       & International & ARSO     & 2022 & Peanut & 5  & 10 & \citet{ARSO2022}             \\
South Sudan       & International & EAC      & 2022 & Peanut & 5  & 10 & \citet{EAC2018}              \\
Spain             & International & EU       & 2002 & Maize  & 2  & 4  & \citet{EC2010}               \\
Spain             & International & EU       & 2022 & Maize  & 2  & 4  & \citet{EC2010}               \\
Spain             & International & EU       & 2002 & Peanut & 2  & 4  & \citet{EC2010}               \\
Spain             & International & EU       & 2022 & Peanut & 2  & 4  & \citet{EC2010}               \\
Sri Lanka         & National      & -        & 2002 & Maize  & NA & 30 & \citet{van2004worldwide}     \\
Sri Lanka         & National      & -        & 2002 & Peanut & NA & 30 & \citet{van2004worldwide}     \\
Sudan             & International & ARSO     & 2022 & Maize  & 5  & 10 & \citet{ARSO2022}             \\
Sudan             & International & ARSO     & 2022 & Peanut & 5  & 10 & \citet{ARSO2022}             \\
Suriname          & National      & -        & 2002 & Maize  & NA & 30 & \citet{van2004worldwide}     \\
Suriname          & National      & -        & 2002 & Peanut & 5  & NA & \citet{van2004worldwide}     \\
Swaziland         & International & ARSO     & 2022 & Maize  & 5  & 10 & \citet{ARSO2022}             \\
Swaziland         & International & ARSO     & 2022 & Peanut & 5  & 10 & \citet{ARSO2022}             \\
Sweden            & International & EU       & 2002 & Maize  & 2  & 4  & \citet{EC2010}               \\
Sweden            & International & EU       & 2022 & Maize  & 2  & 4  & \citet{EC2010}               \\
Sweden            & International & EU       & 2002 & Peanut & 2  & 4  & \citet{EC2010}               \\
Sweden            & International & EU       & 2022 & Peanut & 2  & 4  & \citet{EC2010}               \\
Switzerland       & National      & -        & 2002 & Maize  & 2  & 4  & \citet{van2004worldwide}     \\
Switzerland       & National      & -        & 2022 & Maize  & 2  & 4  & \citet{FDHA2016}             \\
Switzerland       & National      & -        & 2002 & Peanut & 2  & 4  & \citet{van2004worldwide}     \\
Switzerland       & National      & -        & 2022 & Peanut & 2  & 4  & \citet{FDHA2016}             \\
Syria             & National      & -        & 2002 & Peanut & 5  & NA & \citet{van2004worldwide}     \\
Taiwan            & National      & -        & 2002 & Maize  & NA & 15 & \citet{van2004worldwide}     \\
Taiwan            & National      & -        & 2022 & Maize  & 5  & 10 & \citet{TFDA2019}             \\
Taiwan            & National      & -        & 2002 & Peanut & NA & 15 & \citet{van2004worldwide}     \\
Taiwan            & National      & -        & 2022 & Peanut & 2  & 4  & \citet{TFDA2019}             \\
Tanzania          & National      & -        & 2002 & Maize  & 5  & 10 & \citet{van2004worldwide}     \\
Tanzania          & International & ARSO     & 2022 & Maize  & 5  & 10 & \citet{ARSO2022}             \\
Tanzania          & International & EAC      & 2022 & Maize  & 5  & 10 & \citet{EAC2018}              \\
Tanzania          & International & ARSO     & 2022 & Peanut & 5  & 10 & \citet{ARSO2022}             \\
Tanzania          & International & EAC      & 2022 & Peanut & 5  & 10 & \citet{EAC2018}              \\
Thailand          & National      & -        & 2002 & Maize  & NA & 20 & \citet{van2004worldwide}     \\
Thailand          & National      & -        & 2022 & Maize  & NA & 20 & \citet{TACFS2009}            \\
Thailand          & National      & -        & 2002 & Peanut & NA & 20 & \citet{van2004worldwide}     \\
Thailand          & National      & -        & 2022 & Peanut & NA & 20 & \citet{TACFS2014}            \\
Togo              & International & ARSO     & 2022 & Maize  & 5  & 10 & \citet{ARSO2022}             \\
Togo              & International & ARSO     & 2022 & Peanut & 5  & 10 & \citet{ARSO2022}             \\
Tunisia           & National      & -        & 2002 & Maize  & 2  & NA & \citet{van2004worldwide}     \\
Tunisia           & International & ARSO     & 2022 & Maize  & 5  & 10 & \citet{ARSO2022}             \\
Tunisia           & National      & -        & 2002 & Peanut & 2  & NA & \citet{van2004worldwide}     \\
Tunisia           & International & ARSO     & 2022 & Peanut & 5  & 10 & \citet{ARSO2022}             \\
Turkey            & National      & -        & 2002 & Maize  & 2  & 4  & \citet{van2004worldwide}     \\
Turkey            & National      & -        & 2002 & Maize  & 2  & 4  & \citet{van2004worldwide}     \\
Turkey            & National      & -        & 2022 & Maize  & 2  & 4  & \citet{GKGM2011}             \\
Turkey            & National      & -        & 2002 & Peanut & 2  & 4  & \citet{van2004worldwide}     \\
Turkey            & National      & -        & 2002 & Peanut & 5  & 10 & \citet{van2004worldwide}     \\
Turkey            & National      & -        & 2022 & Peanut & 5  & 10 & \citet{GKGM2011}             \\
Uganda            & International & ARSO     & 2022 & Maize  & 5  & 10 & \citet{ARSO2022}             \\
Uganda            & International & EAC      & 2022 & Maize  & 5  & 10 & \citet{EAC2018}              \\
Uganda            & International & ARSO     & 2022 & Peanut & 5  & 10 & \citet{ARSO2022}             \\
Uganda            & International & EAC      & 2022 & Peanut & 5  & 10 & \citet{EAC2018}              \\
UK                & International & EU       & 2002 & Maize  & 2  & 4  & \citet{EC2010}               \\
UK                & International & EU       & 2022 & Maize  & 2  & 4  & \citet{EC2010}               \\
UK                & International & EU       & 2002 & Peanut & 2  & 4  & \citet{EC2010}               \\
UK                & International & EU       & 2022 & Peanut & 2  & 4  & \citet{EC2010}               \\
Ukraine           & National      & -        & 2002 & Maize  & 5  & NA & \citet{van2004worldwide}     \\
Ukraine           & National      & -        & 2022 & Maize  & 2  & 4  & \citet{MOZ2013}              \\
Ukraine           & National      & -        & 2002 & Peanut & 5  & NA & \citet{van2004worldwide}     \\
Ukraine           & National      & -        & 2022 & Peanut & 2  & 4  & \citet{MOZ2013}              \\
United Arab Emirates &
  International &
  - &
  2022 &
  Maize &
  NA &
  4 &
  \citet{van2004worldwide} \\
United Arab Emirates &
  International &
  - &
  2022 &
  Peanut &
  NA &
  4 &
  \citet{van2004worldwide} \\
Uruguay           & International & Mercosur & 2002 & Maize  & NA & 20 & \citet{MERCOSUR2002}         \\
Uruguay           & International & Mercosur & 2002 & Peanut & NA & 20 & \citet{MERCOSUR2002}         \\
USA               & National      & -        & 2002 & Maize  & NA & 20 & \citet{van2004worldwide}     \\
USA               & National      & -        & 2002 & Peanut & NA & 20 & \citet{van2004worldwide}     \\
Venezuela         & National      & -        & 2002 & Maize  & NA & 20 & \citet{van2004worldwide}     \\
Venezuela         & International & Mercosur & 2002 & Maize  & NA & 20 & \citet{MERCOSUR2002}         \\
Venezuela         & National      & -        & 2002 & Peanut & NA & 20 & \citet{van2004worldwide}     \\
Venezuela         & International & Mercosur & 2002 & Peanut & NA & 20 & \citet{MERCOSUR2002}         \\
Vietnam           & National      & -        & 2002 & Maize  & NA & 10 & \citet{van2004worldwide}     \\
Vietnam           & National      & -        & 2022 & Maize  & 2  & 4  & \citet{MOH2011}              \\
Vietnam           & National      & -        & 2002 & Peanut & NA & 10 & \citet{van2004worldwide}     \\
Vietnam           & National      & -        & 2022 & Peanut & 2  & 4  & \citet{MOH2011}              \\
Zambia            & International & ARSO     & 2022 & Maize  & 5  & 10 & \citet{ARSO2022}             \\
Zambia            & International & ARSO     & 2022 & Peanut & 5  & 10 & \citet{ARSO2022}             \\
Zimbabwe          & National      & -        & 2002 & Maize  & 5  & NA & \citet{van2004worldwide}     \\
Zimbabwe          & International & ARSO     & 2022 & Maize  & 5  & 10 & \citet{ARSO2022}             \\
Zimbabwe          & National      & -        & 2002 & Peanut & 5  & NA & \citet{van2004worldwide}     \\
Zimbabwe          & International & ARSO     & 2022 & Peanut & 5  & 10 & \citet{ARSO2022}             \\ 
\hline
\end{longtable}
\begin{minipage}{1.5\textwidth}
%do not draw the footnoterule
\renewcommand{\footnoterule}
\textsuperscript{[a]} Aflatoxin regulation limits are expressed in \textmu g kg \textsuperscript{-1}. 


ARSO = African Organisation for Standardisation;
AU\&NZ = Free trade zone of Australia and New Zealand;
EAC = East African Community;
EACU = Eurasian Customs Union; 
EU = European Union;
GSO =  Gulf Cooperation Council Standardization Organization; 
Mercosur = Southern Common Market;
NA = Not available. \\ 
\end{minipage}
\endgroup
\end{landscape}

%++++++++++++++++++++++++++++++++++++++++++++++++++++++++++++++++++++++++++++++++++++++++

\clearpage

%----------------------------------------------------------------------------------------
\subsection*{Physical-chemical property estimation for Aflatoxins} 
%----------------------------------------------------------------------------------------
\label{Annex:estimation}

Only a limited number of experimental studies have been conducted on the physicochemical properties of AFs, necessitating the use of estimation software applications. The property estimation softwares OCHEM, EPISuite, ACD/Labs and OPERA were used to predict the boiling point (T\textsubscript{b}), melting point (T\textsubscript{m}), vapor pressure  (Log(P\textsubscript{v})), water solubility (Log(c\textsubscript{max,w})), Henry coefficient (Log(K\textsubscript{H})), octanol-air partitioning coefficient (Log(K\textsubscript{OA})), octanol-water partitioning coefficient (Log(K\textsubscript{OW})),   soil absorption coefficient (Log(K\textsubscript{OC})) of the four primary AFs (AFB1, AFB2, AFG1, and AFG2) as well as two major metabolites (AFB2a and AFM1). These estimations were either taken from freely available online databases such as OCHEM, CompTox and Chemspider or manually calculated using the EPI Suite program (version 4.11). The estimations derived from the individual models are presented in table \ref{table:Aflatoxin_estimations}. However, using these models resulted in a high variability in the predicted values from the different calculators. In this regard, \citet{tebes2018demonstration} demonstrated for different chemicals that no individual estimation model outperforms the others, because the performance of the calculators is based on chemical class and the property value. However, the authors found that the geometric mean  and the median of the calculated values from these multiple calculators that use different estimation algorithms are recommended as more reliable estimates of the property value than the value from any single calculator. For that reason the median values were used to derive the input and fate of AFs in soil as described in chapter \ref{subchap:aflatoxins_in_soil} and presented in table \ref{table:Aflatoxin_properties}.

%++++++++++++++++++++++++++++++++++++++++++++++++++++++++++++++++++++++++++++++++++++++++
\begin{landscape}
\begingroup\small
\begin{longtable}[c]{lllll}
\captionsetup{labelfont=bf, justification=justified, singlelinecheck=false, width=1.4\textwidth} 
\caption{Estimations of physicochemical properties and partition coefficients for the four primary aflatoxins (AFB1, AFB2, AFG1, AFG2) and two key metabolites (AFB2a, AFM1). Estimations are derived from models within the EPI Suite software and data from online databases including OCHEM, CompTox, and Chemspider.} 
\\
\label{table:Aflatoxin_estimations}
\\
\hline
\textbf{Substance} & \textbf{Property} & \textbf{Source} & \textbf{Model}                                                                  & \textbf{Value} \\
\hline
\endfirsthead
%
\multicolumn{5}{c}%
{{\bfseries Table \thetable\ continued from previous page}} \\
\hline
\textbf{Substance} & \textbf{Property} & \textbf{Source} & \textbf{Model}                                                                  & \textbf{Value} \\
\hline
\endhead
\hline
\endfoot
%
\endlastfoot
%
%
AFB1      & Log(K\textsubscript{OW})      & OCHEM      & OCHEM logPow (ALogPS 3.0)                                                       & 2.2   \\
          & Log(c\textsubscript{max,w}) & OCHEM      & OCHEM Water solubility (ALogPS 3.0)                                             & -3.8  \\
          & T\textsubscript{m}          & OCHEM      & OCHEM Melting Point (Melting Point prediction (best Estate))                    & 240   \\
          & Log(c\textsubscript{max,w}) & OCHEM      & OCHEM Water solubility (Water solubility model based on logP and Melting Point) & -3.9  \\
          & Log(K\textsubscript{OW})      & OCHEM      & OCHEM logPow (ALOGPS 2.1 logP)                                                  & 1.7   \\
          & Log(c\textsubscript{max,w}) & OCHEM      & OCHEM Water solubility (ALOGPS 2.1 logS)                                        & -3.1  \\
          & Log(K\textsubscript{H})       & CompTox    & OPERA                                                                           & -5.7  \\
          & T\textsubscript{b}          & CompTox    & OPERA                                                                           & 404   \\
          & T\textsubscript{b}          & CompTox    & ACD/Labs                                                                        & 528   \\
          & T\textsubscript{m}          & CompTox    & OPERA                                                                           & 207   \\
          & Log(P\textsubscript{v})       & CompTox    & ACD/Labs                                                                        & -10.5 \\
          & Log(P\textsubscript{v})       & CompTox    & OPERA                                                                           & -8.9  \\
          & Log(c\textsubscript{max,w}) & CompTox    & OPERA                                                                           & -3.0  \\
          & Log(c\textsubscript{max,w}) & CompTox    & ACD/Labs                                                                        & 0.8   \\
          & Log(K\textsubscript{OW})      & CompTox    & OPERA                                                                           & 0.4   \\
          & Log(K\textsubscript{OW})      & CompTox    & ACD/Labs                                                                        & 0.5   \\
          & Log(K\textsubscript{OC})      & CompTox    & OPERA                                                                           & 4.7   \\
          & Log(K\textsubscript{OC})      & Chemspider & ACD/Labs                                                                        & 2.0   \\
          & Log(K\textsubscript{OW})      & EPISuite   & EPISuite KOWWIN v1.69                                                           & 1.2   \\
          & T\textsubscript{b}          & EPISuite   & EPISuite MPBPWIN v1.43                                                          & 474   \\
          & T\textsubscript{m}          & EPISuite   & EPISuite MPBPWIN v1.43                                                          & 200   \\
          & Log(P\textsubscript{v})       & EPISuite   & EPISuite MPBPWIN v1.43                                                          & -8.8  \\
          & Log(c\textsubscript{max,w}) & EPISuite   & EPISuite WSKOW v1.42                                                            & -2.5  \\
          & Log(c\textsubscript{max,w}) & EPISuite   & EPISuite from Fragments (v1.01 est)                                             & -4.7  \\
          & Log(K\textsubscript{H})       & EPISuite   & EPISuite HENRYWIN v3.20 Bond method                                             & -12.9 \\
          & Log(K\textsubscript{H})       & EPISuite   & EPISuite HENRYWIN v3.20 [VP/WSol estimate using EPI values]                     & -9.1  \\
          & Log(K\textsubscript{OC})      & EPISuite   & EPISuite PCKOCWIN v1.66, MCI method                                             & 1.8   \\
          & Log(K\textsubscript{OC})      & EPISuite   & EPISuite PCKOCWIN v1.66, Kow method                                             & 1.8   \\
          \hline
AFB2      & Log(K\textsubscript{OW})      & OCHEM      & OCHEM logPow (ALogPS 3.0)                                                       & 2     \\
          & Log(c\textsubscript{max,w}) & OCHEM      & OCHEM Water solubility (ALogPS 3.0)                                             & -3.9  \\
          & T\textsubscript{m}          & OCHEM      & OCHEM Melting Point (Melting Point prediction (best Estate))                    & 230   \\
          & Log(c\textsubscript{max,w}) & OCHEM      & OCHEM Water solubility (Water solubility model based on logP and Melting Point) & -3.7  \\
          & Log(K\textsubscript{OW})      & OCHEM      & OCHEM logPow (ALOGPS 2.1 logP)                                                  & 1.6   \\
          & Log(c\textsubscript{max,w}) & OCHEM      & OCHEM Water solubility (ALOGPS 2.1 logS)                                        & -2.9  \\
          & Log(K\textsubscript{H})       & CompTox    & OPERA                                                                           & -5.3  \\
          & T\textsubscript{b}          & CompTox    & OPERA                                                                           & 381   \\
          & T\textsubscript{b}          & CompTox    & TEST                                                                            & 455   \\
          & T\textsubscript{b}          & CompTox    & ACD/Labs                                                                        & 521   \\
          & T\textsubscript{m}          & CompTox    & OPERA                                                                           & 287   \\
          & T\textsubscript{m}          & CompTox    & TEST                                                                            & 207   \\
          & Log(P\textsubscript{v})       & CompTox    & ACD/Labs                                                                        & -10.2 \\
          & Log(P\textsubscript{v})       & CompTox    & TEST                                                                            & -9.6  \\
          & Log(P\textsubscript{v})       & CompTox    & OPERA                                                                           & -8.9  \\
          & Log(c\textsubscript{max,w}) & CompTox    & TEST                                                                            & -3.2  \\
          & Log(c\textsubscript{max,w}) & CompTox    & OPERA                                                                           & -2.8  \\
          & Log(c\textsubscript{max,w}) & CompTox    & ACD/Labs                                                                        & 0.8   \\
          & Log(K\textsubscript{OW})      & CompTox    & OPERA                                                                           & 0.5   \\
          & Log(K\textsubscript{OW})      & CompTox    & ACD/Labs                                                                        & 0.4   \\
          & Log(K\textsubscript{OC})      & CompTox    & OPERA                                                                           & 4.7   \\
          & Log(K\textsubscript{OC})      & Chemspider & ACD/Labs                                                                        & 1.9   \\
          & Log(K\textsubscript{OW})      & EPISuite   & EPISuite KOWWIN v1.69                                                           & 1.5   \\
          & T\textsubscript{b}          & EPISuite   & EPISuite MPBPWIN v1.43                                                          & 473   \\
          & T\textsubscript{m}          & EPISuite   & EPISuite MPBPWIN v1.43                                                          & 200   \\
          & Log(P\textsubscript{v})       & EPISuite   & EPISuite MPBPWIN v1.43                                                          & -9.8  \\
          & Log(c\textsubscript{max,w}) & EPISuite   & EPISuite WSKOW v1.42                                                            & -2.7  \\
          & Log(c\textsubscript{max,w}) & EPISuite   & EPISuite from Fragments (v1.01 est)                                             & -4.6  \\
          & Log(K\textsubscript{H})       & EPISuite   & EPISuite HENRYWIN v3.20 Bond method                                             & -14.5 \\
          & Log(K\textsubscript{H})       & EPISuite   & EPISuite HENRYWIN v3.20 [VP/WSol estimate using EPI values]                     & -12.9 \\
          & Log(K\textsubscript{OC})      & EPISuite   & EPISuite PCKOCWIN v1.66, MCI method                                             & 1.8   \\
          & Log(K\textsubscript{OC})      & EPISuite   & EPISuite PCKOCWIN v1.66, Kow method                                             & 1.9   \\
          \hline
AFG1      & Log(K\textsubscript{OW})      & OCHEM      & OCHEM logPow (ALogPS 3.0)                                                       & 2.2   \\
          & Log(c\textsubscript{max,w}) & OCHEM      & OCHEM Water solubility (ALogPS 3.0)                                             & -3.5  \\
          & T\textsubscript{m}          & OCHEM      & OCHEM Melting Point (Melting Point prediction (best Estate))                    & 230   \\
          & Log(c\textsubscript{max,w}) & OCHEM      & OCHEM Water solubility (Water solubility model based on logP and Melting Point) & -3.8  \\
          & Log(K\textsubscript{OW})      & OCHEM      & OCHEM logPow (ALOGPS 2.1 logP)                                                  & 1.8   \\
          & Log(c\textsubscript{max,w}) & OCHEM      & OCHEM Water solubility (ALOGPS 2.1 logS)                                        & -2.9  \\
          & Log(K\textsubscript{H})       & CompTox    & OPERA                                                                           & -8.2  \\
          & T\textsubscript{b}          & CompTox    & OPERA                                                                           & 403   \\
          & T\textsubscript{b}          & CompTox    & TEST                                                                            & 497   \\
      & T\textsubscript{b}          & CompTox    & ACD/Labs                                                                        & 612   \\
          & T\textsubscript{m}          & CompTox    & OPERA                                                                           & 245   \\
          & T\textsubscript{m}          & CompTox    & TEST                                                                            & 228   \\
          & Log(P\textsubscript{v})       & CompTox    & ACD/Labs                                                                        & -14.2 \\
          & Log(P\textsubscript{v})       & CompTox    & OPERA                                                                           & -10.0 \\
          & Log(c\textsubscript{max,w}) & CompTox    & TEST                                                                            & -3.1  \\
          & Log(c\textsubscript{max,w}) & CompTox    & OPERA                                                                           & -6.0  \\
          & Log(c\textsubscript{max,w}) & CompTox    & ACD/Labs                                                                        & 0.8   \\
          & Log(K\textsubscript{OW})      & CompTox    & OPERA                                                                           & 1.8   \\
          & Log(K\textsubscript{OW})      & CompTox    & ACD/Labs                                                                        & -0.2  \\
          & Log(K\textsubscript{OC})      & CompTox    & OPERA                                                                           & 5.3   \\
          & Log(K\textsubscript{OC})      & Chemspider & ACD/Labs                                                                        & 1.8   \\
          & Log(K\textsubscript{OW})      & EPISuite   & EPISuite KOWWIN v1.69                                                           & 0.5   \\
          & T\textsubscript{b}          & EPISuite   & EPISuite MPBPWIN v1.43                                                          & 511   \\
          & T\textsubscript{m}          & EPISuite   & EPISuite MPBPWIN v1.43                                                          & 218   \\
          & Log(P\textsubscript{v})       & EPISuite   & EPISuite MPBPWIN v1.43                                                          & -10.2 \\
          & Log(c\textsubscript{max,w}) & EPISuite   & EPISuite WSKOW v1.42                                                            & -2.0  \\
          & Log(c\textsubscript{max,w}) & EPISuite   & EPISuite from Fragments (v1.01 est)                                             & -4.2  \\
          & Log(K\textsubscript{H})       & EPISuite   & EPISuite HENRYWIN v3.20 Bond method                                             & -12.3 \\
          & Log(K\textsubscript{H})       & EPISuite   & EPISuite HENRYWIN v3.20 [VP/WSol estimate using EPI values]                     & -14.1 \\
          & Log(K\textsubscript{OC})      & EPISuite   & EPISuite PCKOCWIN v1.66, MCI method                                             & 1.9   \\
          & Log(K\textsubscript{OC})      & EPISuite   & EPISuite PCKOCWIN v1.66, Kow method                                             & 1.1   \\
          \hline
AFG2      & Log(K\textsubscript{OW})      & OCHEM      & OCHEM logPow (ALogPS 3.0)                                                       & 1.6   \\
          & Log(c\textsubscript{max,w}) & OCHEM      & OCHEM Water solubility (ALogPS 3.0)                                             & -3.5  \\
          & T\textsubscript{m}          & OCHEM      & OCHEM Melting Point (Melting Point prediction (best Estate))                    & 220   \\
          & Log(c\textsubscript{max,w}) & OCHEM      & OCHEM Water solubility (Water solubility model based on logP and Melting Point) & -3.6  \\
          & Log(K\textsubscript{OW})      & OCHEM      & OCHEM logPow (ALOGPS 2.1 logP)                                                  & 1.6   \\
          & Log(c\textsubscript{max,w}) & OCHEM      & OCHEM Water solubility (ALOGPS 2.1 logS)                                        & -2.8  \\
          & Log(K\textsubscript{H})       & CompTox    & OPERA                                                                           & -5.8  \\
          & T\textsubscript{b}          & CompTox    & OPERA                                                                           & 401   \\
          & T\textsubscript{b}          & CompTox    & ACD/Labs                                                                        & 603   \\
          & T\textsubscript{m}          & CompTox    & OPERA                                                                           & 189   \\
          & Log(P\textsubscript{v})       & CompTox    & ACD/Labs                                                                        & -13.7 \\
          & Log(P\textsubscript{v})       & CompTox    & OPERA                                                                           & -9.9  \\
          & Log(c\textsubscript{max,w}) & CompTox    & OPERA                                                                           & -1.5  \\
          & Log(c\textsubscript{max,w}) & CompTox    & ACD/Labs                                                                        & 0.8   \\
          & Log(K\textsubscript{OW})      & CompTox    & OPERA                                                                           & 0.5   \\
          & Log(K\textsubscript{OW})      & CompTox    & ACD/Labs                                                                        & -0.3  \\
          & Log(K\textsubscript{OC})      & CompTox    & OPERA                                                                           & 4.3   \\
          & Log(K\textsubscript{OC})      & Chemspider & ACD/Labs                                                                        & 1.7   \\
          & Log(K\textsubscript{OW})      & EPISuite   & EPISuite KOWWIN v1.69                                                           & 0.7   \\
          & T\textsubscript{b}          & EPISuite   & EPISuite MPBPWIN v1.43                                                          & 510   \\
          & T\textsubscript{m}          & EPISuite   & EPISuite MPBPWIN v1.43                                                          & 217   \\
          & Log(P\textsubscript{v})       & EPISuite   & EPISuite MPBPWIN v1.43                                                          & -9.9  \\
          & Log(c\textsubscript{max,w}) & EPISuite   & EPISuite WSKOW v1.42                                                            & -2.2  \\
          & Log(c\textsubscript{max,w}) & EPISuite   & EPISuite from Fragments (v1.01 est)                                             & -4.1  \\
          & Log(K\textsubscript{H})       & EPISuite   & EPISuite HENRYWIN v3.20 Bond method                                             & -14.0 \\
          & Log(K\textsubscript{H})       & EPISuite   & EPISuite HENRYWIN v3.20 [VP/WSol estimate using EPI values]                     & -13.5 \\
          & Log(K\textsubscript{OC})      & EPISuite   & EPISuite PCKOCWIN v1.66, MCI method                                             & 1.9   \\
          & Log(K\textsubscript{OC})      & EPISuite   & EPISuite PCKOCWIN v1.66, Kow method                                             & 1.2   \\
          \hline
AFB2a     & Log(K\textsubscript{OW})      & OCHEM      & OCHEM logPow (ALogPS 3.0)                                                       & 1.0   \\
          & Log(c\textsubscript{max,w}) & OCHEM      & OCHEM Water solubility (ALogPS 3.0)                                             & -3.2  \\
          & T\textsubscript{m}          & OCHEM      & OCHEM Melting Point (Melting Point prediction (best Estate))                    & 240   \\
          & Log(c\textsubscript{max,w}) & OCHEM      & OCHEM Water solubility (Water solubility model based on logP and Melting Point) & -3.1  \\
          & Log(K\textsubscript{OW})      & OCHEM      & OCHEM logPow (ALOGPS 2.1 logP)                                                  & 1.0   \\
          & Log(c\textsubscript{max,w}) & OCHEM      & OCHEM Water solubility (ALOGPS 2.1 logS)                                        & -2.4  \\
          & Log(K\textsubscript{H})       & CompTox    & OPERA                                                                           & -8.2  \\
          & T\textsubscript{b}          & CompTox    & OPERA                                                                           & 389   \\
          & T\textsubscript{b}          & CompTox    & TEST                                                                            & 485   \\
          & T\textsubscript{b}          & CompTox    & ACD/Labs                                                                        & 575   \\
          & T\textsubscript{m}          & CompTox    & OPERA                                                                           & 187   \\
          & T\textsubscript{m}          & CompTox    & TEST                                                                            & 233   \\
          & Log(P\textsubscript{v})       & CompTox    & ACD/Labs                                                                        & -13.3 \\
          & Log(P\textsubscript{v})       & CompTox    & TEST                                                                            & -12.2 \\
          & Log(P\textsubscript{v})       & CompTox    & OPERA                                                                           & -8.5  \\
          & Log(c\textsubscript{max,w}) & CompTox    & TEST                                                                            & -3.1  \\
          & Log(c\textsubscript{max,w}) & CompTox    & OPERA                                                                           & -1.0  \\
          & Log(K\textsubscript{OW})      & CompTox    & OPERA                                                                           & -215  \\
          & Log(K\textsubscript{OW})      & CompTox    & ACD/Labs                                                                        & -0.5  \\
          & Log(K\textsubscript{OC})      & CompTox    & OPERA                                                                           & 4.3   \\
          & Log(K\textsubscript{OC})      & Chemspider & ACD/Labs                                                                        & 1.5   \\
          & Log(K\textsubscript{OW})      & EPISuite   & EPISuite KOWWIN v1.69                                                           & -0.4  \\
          & T\textsubscript{b}          & EPISuite   & EPISuite MPBPWIN v1.43                                                          & 508.8 \\
          & T\textsubscript{m}          & EPISuite   & EPISuite MPBPWIN v1.43                                                          & 216.7 \\
          & Log(P\textsubscript{v})       & EPISuite   & EPISuite MPBPWIN v1.43                                                          & >0.1  \\
          & Log(c\textsubscript{max,w}) & EPISuite   & EPISuite WSKOW v1.42                                                            & -0.7  \\
          & Log(c\textsubscript{max,w}) & EPISuite   & EPISuite from Fragments (v1.01 est)                                             & -2.8  \\
          & Log(K\textsubscript{H})       & EPISuite   & EPISuite HENRYWIN v3.20 Bond method                                             & -19.0 \\
          & Log(K\textsubscript{H})       & EPISuite   & EPISuite HENRYWIN v3.20 [VP/WSol estimate using EPI values]                     & -17.2 \\
          & Log(K\textsubscript{OC})      & EPISuite   & EPISuite PCKOCWIN v1.66, MCI method                                             & 1     \\
          & Log(K\textsubscript{OC})      & EPISuite   & EPISuite PCKOCWIN v1.66, Kow method                                             & 0.4   \\
          \hline
AFM1      & Log(K\textsubscript{OW})      & OCHEM      & OCHEM logPow (ALogPS 3.0)                                                       & 1.5   \\
          & Log(c\textsubscript{max,w}) & OCHEM      & OCHEM Water solubility (ALogPS 3.0)                                             & -3.5  \\
          & T\textsubscript{m}          & OCHEM      & OCHEM Melting Point (Melting Point prediction (best Estate))                    & 230   \\
          & Log(c\textsubscript{max,w}) & OCHEM      & OCHEM Water solubility (Water solubility model based on logP and Melting Point) & -3.3  \\
          & Log(K\textsubscript{OW})      & OCHEM      & OCHEM logPow (ALOGPS 2.1 logP)                                                  & 1.2   \\
          & Log(c\textsubscript{max,w}) & OCHEM      & OCHEM Water solubility (ALOGPS 2.1 logS)                                        & -2.5  \\
          & Log(K\textsubscript{H})       & CompTox    & OPERA                                                                           & -8.7  \\
          & T\textsubscript{b}          & CompTox    & OPERA                                                                           & 409   \\
          & T\textsubscript{b}          & CompTox    & ACD/Labs                                                                        & 644   \\
          & T\textsubscript{m}          & CompTox    & OPERA                                                                           & 170   \\
          & Log(P\textsubscript{v})       & CompTox    & ACD/Labs                                                                        & -16.8 \\
          & Log(P\textsubscript{v})       & CompTox    & OPERA                                                                           & -8.6  \\
          & Log(c\textsubscript{max,w}) & CompTox    & OPERA                                                                           & -1.2  \\
          & Log(c\textsubscript{max,w}) & CompTox    & ACD/Labs                                                                        & 0.8   \\
          & Log(K\textsubscript{OW})      & CompTox    & OPERA                                                                           & -0.2  \\
          & Log(K\textsubscript{OW})      & CompTox    & ACD/Labs                                                                        & -0.3  \\
          & Log(K\textsubscript{OC})      & CompTox    & OPERA                                                                           & 4.3   \\
          & Log(K\textsubscript{OC})      & Chemspider & ACD/Labs                                                                        & 1.5   \\
          & Log(K\textsubscript{OW})      & EPISuite   & EPISuite KOWWIN v1.69                                                           & -0.3  \\
          & T\textsubscript{b}          & EPISuite   & EPISuite MPBPWIN v1.43                                                          & 502   \\
          & T\textsubscript{m}          & EPISuite   & EPISuite MPBPWIN v1.43                                                          & 214   \\
          & Log(P\textsubscript{v})       & EPISuite   & EPISuite MPBPWIN v1.43                                                          & -11.8 \\
          & Log(c\textsubscript{max,w}) & EPISuite   & EPISuite WSKOW v1.42                                                            & -0.9  \\
          & Log(c\textsubscript{max,w}) & EPISuite   & EPISuite from Fragments (v1.01 est)                                             & -2.7  \\
          & Log(K\textsubscript{H})       & EPISuite   & EPISuite HENRYWIN v3.20 Bond method                                             & -17.3 \\
          & Log(K\textsubscript{H})       & EPISuite   & EPISuite HENRYWIN v3.20 [VP/WSol estimate using EPI values]                     & -16.8 \\
          & Log(K\textsubscript{OC})      & EPISuite   & EPISuite PCKOCWIN v1.66, MCI method                                             & 1     \\
          & Log(K\textsubscript{OC})      & EPISuite   & EPISuite PCKOCWIN v1.66, Kow method                                             & 0.5  \\
          \hline
\end{longtable}
\begin{minipage}{1.4\textwidth}
%do not draw the footnoterule
\renewcommand{\footnoterule}{}
T\textsubscript{b} = Boiling point; T\textsubscript{m} = Melting point; Log(P\textsubscript{v}) = Vapor pressure, logarithmic scale; Log(c\textsubscript{max,w}) = Water solubility, logarithmic scale; Log(K\textsubscript{OA}) = Octanol-Air-partitioning coefficient, logarithmic scale; Log(K\textsubscript{OW}) = Octanol-Water-partitioning coefficient, logarithmic scale; Log(K\textsubscript{H}) = Henry coefficent, logarithmic scale; Log(K\textsubscript{OC}) = Soil absorption coefficient, logarithmic scale. 
\end{minipage}
\endgroup
\end{landscape} 
%++++++++++++++++++++++++++++++++++++++++++++++++++++++++++++++++++++++++++++++++++++++++

\clearpage
