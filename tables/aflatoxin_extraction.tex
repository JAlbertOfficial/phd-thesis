\begin{table}[t!]
\footnotesize
%\centering
\begin{adjustwidth}{-0.04\textwidth}{0.15\textwidth}% adjust the L and R margins by 1 inch
\begin{threeparttable}
\hfuzz = 100pt
\captionsetup{labelfont=bf, justification=justified, singlelinecheck=false, width=1\textwidth} 
\caption{Previously described analytical procedures for the extraction of AFs from soil matrices.} \label{table:Aflatoxin_extraction}
\begin{tabular}{lllllll}
\toprule
\textbf{\begin{tabular}[c]{@{}l@{}}Extraction\\ technique\end{tabular}} & \textbf{\begin{tabular}[c]{@{}l@{}}Extraction \\ solvents\end{tabular}}   & \textbf{\begin{tabular}[c]{@{}l@{}}Extraction\\ procedure\end{tabular}}   & \textbf{\begin{tabular}[c]{@{}l@{}}Soil \\ type\end{tabular}} & \textbf{\begin{tabular}[c]{@{}l@{}}AFs range\\ (µg kg-1)\end{tabular}} & \textbf{\begin{tabular}[c]{@{}l@{}}Recovery \\ (\%)\end{tabular}} & \textbf{Reference}  \\
\midrule
SLE & \begin{tabular}[c]{@{}l@{}}\ce{H2O}:EtOAc \\ (1:3)\end{tabular}& \begin{tabular}[c]{@{}l@{}}overnight \\ shaking\end{tabular}  & silt loam & 10 \textsuperscript{1}   & NA & \citet{accinelli2008aspergillus}\\
SLE & ACE   & \begin{tabular}[c]{@{}l@{}}30 min \\ shaking\end{tabular} & silt loam & 10 \textsuperscript{4}   & 18 & \citet{angle1980decomposition}\\
SLE & \begin{tabular}[c]{@{}l@{}}\ce{CHCl3}, MeOH, \\ \ce{CHCl3}:MeOH \\ (80:20)\end{tabular} & NA& loam soil & 10 \textsuperscript{4} - 10 \textsuperscript{5} & \textless{}1   & \citet{mertz1981absorption}\\
\multirow{4}{*}{SLE}& \multirow{4}{*}{ACE}  & \multirow{4}{*}{\begin{tabular}[c]{@{}l@{}}saturation \\ with \ce{H2O}, \\ 5 min \\ blending\end{tabular}} & silt loam,& \multirow{4}{*}{10 \textsuperscript{3}}  & \multirow{4}{*}{70}& \multirow{4}{*}{\citet{goldberg1985aflatoxin}} \\
&   &   & sandy loam,   &   && \\
&   &   & clay loam,&   && \\
&   &   & silty clay loam   &   && \\
SFE & \begin{tabular}[c]{@{}l@{}}MeCN + \\ 2\% AcOH\end{tabular}& \begin{tabular}[c]{@{}l@{}}15 min \\ static time\end{tabular} & silt loam & 10 \textsuperscript{3}   & 72 & \citet{starr2008supercritical}   
\\
\bottomrule
\end{tabular}
\begin{tablenotes}[flushleft]
\setlength\labelsep{0pt}
\footnotesize 
%\itshape
\item SLE = Solid-liquid-extraction; SFE = Supercritical-fluid-extraction; EtOAc = Ethyl acetate; ACE = Acetone; MeOH = Methanol; AcOH = Acetic acid. 
\end{tablenotes}
\end{threeparttable}
\end{adjustwidth}

\end{table}
