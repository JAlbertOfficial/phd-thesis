    \begin{table}[ht!]
    \footnotesize
        \begin{threeparttable}        
        \captionsetup{labelfont=bf, justification=justified, singlelinecheck=false, width=1\textwidth} 
        \caption{Physico-chemical properties and partition coefficients for the four main aflatoxins (AFB1, AFB2, AFG1, AFG2) and two main metabolites (AFB2a, AFM1). Property values represent the median of estimates from various models implemented in physical-chemical property estimation software (including OCHEM, EPISuite, ACD/Labs, and OPERA) \citep{tebes2018demonstration}. The individual estimates and models are given in Chapter \ref{Annex_chap1}.} 
\begin{tabular}{llllllll}  
	\toprule
	\textbf{Property}           & \textbf{Unit}                                    & \textbf{AFB1} & \textbf{AFB2} & \textbf{AFG1} & \textbf{AFG2} & \textbf{AFB2a} & \textbf{AFM1} \\
	\midrule
	Formula                     & -                                                & \ce{C17H12O6} & \ce{C17H14O6} & \ce{C17H12O7} & \ce{C17H14O7} & \ce{C17H14O7}  & \ce{C17H12O7} \\
	M                           & g mol\textsuperscript{-1}                        & 312.27        & 314.29        & 328.28        & 330.29        & 330.29         & 328.28        \\
	M\textsubscript{mi}         & g mol\textsuperscript{-1}                        & 312.06        & 314.08        & 328.06        & 330.07        & 330.07         & 328.06        \\
	HBA                         & -                                                & 6             & 6             & 7             & 7             & 7              & 7             \\
	HBD                         & -                                                & 0             & 0             & 0             & 0             & 1              & 1             \\
	
	
	T\textsubscript{b}          & °C                                              & 474           & 472           & 511           & 510           & 509            & 502           \\
	T\textsubscript{m}          & °C                                              & 207           & 230           & 230           & 217           & 217            & 214           \\
	log(P\textsubscript{v})     & mmHg                                             & -8.9          & -9.8          & -10.2         & -9.9          & -8.5           & -11.8         \\
	Log(c\textsubscript{max,w}) & mol L\textsuperscript{-1}                        & -3.1          & -2.9          & -3.5          & -2.8          & -2.6           & -2.5          \\
	Log(K\textsubscript{OW})    & -                                                & 1.2           & 1.4           & 1.8           & 0.7           & -0.4           & -0.2          \\
	Log(K\textsubscript{H})     & atm m\textsuperscript{3} mol\textsuperscript{-1} & -9.1          & -12.9         & -12.3         & -13.5         & -17.2          & -16.8         \\
	Log(K\textsubscript{OC})    & L/kg                                             & 1.9           & 1.9           & 1.8           & 1.8           & 1.2            & 1.3           \\
	\bottomrule
\end{tabular}
\label{table:Aflatoxin_properties}
            \begin{tablenotes}[flushleft]
                \setlength\labelsep{0pt}
                \footnotesize 
                %\itshape
                \item Formula = Empiric formula; M = Molar mass; M\textsubscript{mi} = Monoisotopic mass; HBA = Hydrogen Bond Acceptor Count; HBD = Hydrogen Bond Donor Count; T\textsubscript{b} = Boiling point; T\textsubscript{m} = Melting point; Log(P\textsubscript{v}) = Vapor pressure, logarithmic scale; Log(c\textsubscript{max,w}) = Water solubility, logarithmic scale; Log(K\textsubscript{OA}) = Octanol-Air-partitioning coefficient, logarithmic scale; Log(K\textsubscript{OW}) = Octanol-Water-partitioning coefficient, logarithmic scale; Log(K\textsubscript{H}) = Henry coefficent, logarithmic scale; Log(K\textsubscript{OC}) = Soil absorption coefficient, logarithmic scale. 
            \end{tablenotes}
        \end{threeparttable}
       % \end{adjustwidth}
    \end{table}