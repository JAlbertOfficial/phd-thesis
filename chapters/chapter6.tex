%########################################################################################
\chapter{Synthesis and Conclusions} \label{Chapter6}  
%########################################################################################

\clearpage

%========================================================================================
\section{Challenges in Aflatoxin Analysis and Monitoring in Soil} \label{subchap:synthesis_analysis}
%========================================================================================

Understanding the occurrence, fate, and impact of aflatoxins in the soil environment requires a comprehensive and systematic approach that begins with the development of robust analytical methods. This is essential to accurately represent the actual situation of residual aflatoxin concentrations. Four different factors can be considered as major obstacles to the development of analytical methods and sampling strategies: (1) The soil matrix exhibits an inherent complexity characterized by strong interactions between aflatoxins and certain soil fractions (Chapter \ref{subchap:dissipation} and \ref{subchap:analysis}). This phenomenon is similar to that observed in the analysis of various organic pollutants \citep{trellu2016removal}. (2) Chromatographic separation is a critical factor in aflatoxin analysis, especially in the context of rapid analysis required for extensive field campaigns. Streamlining the analytical process is essential in such scenarios and requires minimizing or eliminating tedious and costly purification steps while ensuring the effective separation of aflatoxins from co-extracted matrix interferences. (3) The spatial and temporal heterogeneity in the occurrence of mycotoxins in soils further complicates the analysis. This variability is closely related to the various pathways by which aflatoxins can enter the soil environment and the diverse soil processes that determine their fate (Chapter \ref{subchap:entry}). This heterogeneity is consistent with monitoring campaigns at the food and feed commodity level \citep{miraglia2005role}. (4) Soil is a living matrix in which soil processes such as degradation can play an important role. In this context, appropriate sampling strategies may include recommendations for transport and storage to reliably assess environmental concentrations \citep{wagner1995basic}.

%----------------------------------------------------------------------------------------
\subsection{Overcoming Soil--Aflatoxin Interactions in the Extraction of Aflatoxin from Soil} 
%----------------------------------------------------------------------------------------

As described in Chapter \ref{subchap:aflatoxins_in_soil}, a limited number of studies have investigated the occurrence and biosynthesis, sorption and leaching, and degradation and mineralization of aflatoxins in soil \citep{accinelli2008aspergillus, goldberg1985aflatoxin, angle1980decomposition, angle1986aflatoxin}. However, interpretation of these results is hampered by insufficient extraction recoveries, the use of spike concentrations well above natural concentrations, and the lack of systematic validation of analytical methods (Chapter \ref{subchap:aflatoxins_in_soil}, Table \ref{table:Aflatoxin_extraction}). 


This obstacle can be attributed to the complicated and heterogeneous nature of the soil as a matrix and the strong adsorption affinity of the AFs at the binding sites in the soil (Chapter \ref{subchap:dissipation}). To overcome these methodological challenges, it was essential to disrupt the chemical interactions between the soil matrix and the AFs to facilitate their transition to the liquid phase. The first step, therefore, was to identify the specific soil properties that are primarily responsible for this pronounced interaction and to clarify what underlies these interactions. In this context, as elucidated in Chapter \ref{subchap:dissipation} and \ref{subchap:analysis}, it was found that the strong sorption affinity of AFs to soil can largely be attributed to clay minerals.  While studies by \citet{schenzel2012experimentally} and \citet{van2006vitro} demonstrated the interaction of AFs with organic matter, leaching experiments conducted by \citet{goldberg1985aflatoxin} on a range of structurally diverse soils underscored that clay content is the main determinant of the particular strong interaction between AF and soil. \citet{kang2016understanding} further demonstrated that electron donor-acceptor interactions between the two electron-rich carbonyl groups within the coumarin structure of AF and positively charged species located on the negatively charged surfaces of clay minerals (e.g., \ce{H+} for illite and Ca\textsuperscript{2+} for smectite) are primarily responsible for the strong sorption of AFs onto clays. 

In this work, the development and validation of a simple and reliable analytical method for the quantification of aflatoxins in soil and plant-based food matrices is described (Chapter \ref{Chapter2}). The presented approach involved the utilization of an efficient extraction solvent mixture comprising acetonitrile and water, coupled with an ultrasonication step. Recoveries of 78 to 92\% were obtained with the presented method, allowing reliable determination at environmentally relevant concentrations of 0.5 to 20 \textmu{}g kg\textsuperscript{-1}. This is the first time a successful solvent extraction method has been presented for the quantitative analysis of AFs in both soil and food matrices. So far, only one method has achieved a satisfactory recovery of 72\%, using the much more complicated and expensive supercritical fluid extraction approach \citep{starr2008supercritical}. Acetonitrile, a monopolar solvent with H-bond acceptor properties, exhibited similar characteristics to the carbonyl groups in the coumarin structure of aflatoxins and consequently displaced the aflatoxins from the H-bond sites on the cations located on the negatively charged surfaces of clay mineral substrates. It is noteworthy that previous studies by \citet{madden1993preliminary} using a solvent mixture of similar composition yielded only trace amounts of aflatoxins, probably due to the absence of an ultrasonic step. Ultrasonic treatment is known to reduce the size of soil agglomerates and clay minerals, thereby increasing the surface area \citep{lesueur2008comparison}. This property makes it a preferred step in the extraction process of organic pollutants from the soil \citep{bossio2008application}. However, the limited selectivity of ultrasonic treatment results in the simultaneous extraction of a high load of matrix components along with the analytes, substantially compromising the analytical performance of the separation and detection method. 

%----------------------------------------------------------------------------------------
\subsection{Resolving Challenges in Separation and Detection Arising from Matrix Interference} 
%----------------------------------------------------------------------------------------

Both LC-MS and HPLC-FLD were found to be suitable for analysis using the method presented in Chapter \ref{Chapter2}, although there were problems with co-extracted matrix components. Quantitative analysis using MS techniques with electrospray ionization (ESI) or atmospheric pressure chemical ionization (APCI) can be significantly affected by the occurrence of ion suppression or enrichment due to the high ion loading in soil and sediment samples \citep{trufelli2011overview}. Therefore, the LC-MS approach experienced signal reductions of up to -25\% and -54\% for soil and food samples, respectively. Consequently, sample purification techniques such as immunoaffinity chromatography (IAC) or solid-phase extraction (SPE), as well as matrix effect compensation strategies like matrix-matched calibration (MMC) and stable isotope dilution assays (SIDA), would be necessary \citep{shephard2009aflatoxin, razzazi2011sample}. In the final procedure presented in Chapter \ref{Chapter2}, an MMC approach was chosen instead of using costly SIDA or purification steps. However, the use of MMC was only possible because analyte-free samples were available for the matrices under investigation, making the more expensive methods necessary when a sample blank is unavailable.


In contrast, HPLC-FLD exhibited minimal coeluting interferences and negligible matrix effects, rendering it more suitable for routine analysis. To overcome interferences during separation, an unconventional mobile phase composed of a mixture of water, methanol, and acetonitrile (in a ratio of 72:20:8, v/v/v) was employed, offering a compromise between separation efficiency and speed. The relatively high water content of 72\% was essential to sufficiently separate interferences from the analytes, albeit at the expense of longer run times. Similarly, a relatively high methanol content was required to achieve an adequate separation between aflatoxins AFG1 and AFB2, leading to extended runtime compared to higher acetonitrile contents. In addition, the HPLC-FLD showed a sensitivity in terms of limit of detection and quantification comparable to LC-MS, which is normally known for its better sensitivity. The sensitivity of the HPLC-FLD was achieved by injecting a high volume of 100 µl, facilitated by the on-column focusing technique \citep{vissers1996optimised, mills1997assessment, groskreutz2015quantitative} in which the sample was prepared in a weaker solvent (80:20, water/methanol) than the mobile phase (72:20:8, water/methanol/acetonitrile). The absence of a purification step and the ability to use HPLC-FLD significantly reduced the labor and cost of the analytical process. Therefore, this method is particularly promising for routine analysis in regions where aflatoxin levels may be a health concern and require continuous assessment of environmental contamination. In addition, its simplicity and rapidity offer the potential for capacity building, as it does not require complex and expensive analytical equipment. This is particularly beneficial in regions affected by aflatoxin contamination, especially in Sub-Saharan Africa, where lack of advanced analytical equipment and financial constraints can be limiting factors \citep{gnonlonfin2013review}.

%----------------------------------------------------------------------------------------
\subsection{Representative Field Sampling in the Face of Spatial Heterogeneity, Seasonality and Aflatoxin Instability} 
%----------------------------------------------------------------------------------------

In environmental monitoring of aflatoxins in soil, the challenge is not only to extract aflatoxins from the soil matrix but also to obtain a truly representative soil sample for the entire field or a specific sampling unit. In my thesis (Chapter \ref{Chapter3}), a comprehensive field study is presented, which aims to investigate the occurrence of AFs in soils and identify potential influences of agricultural practices, soil depth, and field location. However, no aflatoxins were detected in the soil samples, despite the presence of aflatoxins in maize samples grown in the same field and toxigenic fungi were identified in the soil samples. This inconsistency led to a deeper investigation of the underlying factors. 


The inherent heterogeneity of agricultural soils, both in terms of their spatial distribution across fields and their vertical profile, together with the concentrated colonization of grain-rich plant residues by toxigenic fungi \citep{horn2003ecology}, may result in localized areas of elevated aflatoxin contamination \citep{accinelli2008aspergillus}, with the potential for variation in mycotoxin concentrations even within small regions \citep{kenngott2022fusarium}. To address this small-scale heterogeneity, a sophisticated approach of collecting multiple individual samples from a fine-mesh network of sampling sites within specific sampling clusters at two depths (topsoil and subsoil) and two positions (between plants and inter-row). This methodology had already proven successful in detecting \textit{Fusarium} toxins, including nivalenol and deoxynivalenol, in maize field soils in Germany \citep{kenngott2022fusarium}. Further, the analytical procedure employed for the Kenyan soil extracts adhered closely to the method detailed by Kenngott et al. (2022) and yet yielded negative results for the presence of \textit{Fusarium} toxins, including nivalenol, deoxynivalenol, 15-acetyl-deoxynivalenol, and zearalenone.  It's noteworthy that \textit{Fusarium} fungi were indeed detected in these Kenyan soils. 


Considering the above factors, it seems unlikely that the absence of aflatoxins in the soil samples was due to an inadequate sampling procedure. Rather, it is plausible that the aflatoxins dissipated during the long storage and transport periods, which spanned approximately 2.5 months from the initial sampling to the analysis phase. This phenomenon was experimentally investigated in Chapter 4 of the study, which revealed that AFB1 was rapidly degraded in two reference soils, with half-lives ranging from 20 to 65 days, depending on various environmental conditions, including UV light, microbial degradation, and sterile conditions.


In summary, the research underscores the importance of frequent timed soil sampling throughout the corn growing cycle in conjunction with analyses immediately after sampling or proper storage of samples to minimize potential dissipation during transport and storage. These principles align with established practices for monitoring several other soil microbial parameters, including phospholipid-derived fatty acids \citep{petersen1994effects, veum2019phospholipid} and soil microbial biomass carbon \citep{vcernohlavkova2009variability, stenberg1998microbial}, as well as xenobiotics such as pesticides \citep{lehotay2015sampling}. However, it is critical to recognize that these practices extend beyond the laboratory and into international collaborative projects, such as the project conducted with Kenya in this study. As part of such collaborations, capacity building, networking, and the establishment of local laboratory infrastructure are essential. These efforts would enable timely analysis of samples, minimize errors, and ensure accurate results in determining residual concentrations of mycotoxins in the soil environment. 

%========================================================================================
\section{Environmental Relevance of Aflatoxins in the Soil Environment}
%========================================================================================

When evaluating the environmental relevance of a substance, several aspects must be taken into account, encompassing (1) the extent of the substance's presence in the environment and the factors influencing its occurrence, (2) the environmental fate of the substance in the environment, including the processes it undergoes and the factors influencing these processes, and (3) the consequences of the substance's presence on organisms in the environment and the associated functions. In a review by \citet{fouche2020aflatoxins}, potential ecological consequences associated with aflatoxins occurrence in soil are explored, although there is currently limited empirical evidence available. Furthermore, various reviews \citep{fouche2020aflatoxins, elmholt2008mycotoxins, juraschek2022mycotoxins} have theoretically elucidated how aflatoxins can enter the soil and how anthropogenic activities may lead to additional aflatoxin inputs, potentially disrupting the balance between depletion and accumulation in soil (Chapter \ref{subchap:aflatoxins_in_soil}). However, experimental studies directly investigating the extent and processes of aflatoxin occurrence in soil have been notably scarce. As indicated by \citet{elmholt2008mycotoxins} and \citet{abbas2009ecology}, one of the primary reasons for this scarcity lies in the unresolved methodological challenges associated with detecting aflatoxins in soil, which are essential for addressing these research objectives. In this context, the present thesis has successfully addressed some of these methodological issues and provided potential solutions, as detailed in Chapter \ref{subchap:synthesis_analysis}. These developments have opened the door to further investigations into the occurrence, fate, and implications of aflatoxins in soil.

%----------------------------------------------------------------------------------------
\subsection{Aflatoxin Occurence in Agricultural Soils}
%----------------------------------------------------------------------------------------

Knowledge on the presence of AFs in soils and crop residues remains limited, and little information is available on the extent and causes. A notable contribution in this field was made by \citet{accinelli2008aspergillus}, who demonstrated that aflatoxins are synthesized in the soil at varying levels i.e. in the range of 10\textsuperscript{2} (cobs containing grain), 10\textsuperscript{0} (leaves, stalks and cobs without grain) and 10\textsuperscript{-1} \textmu g kg\textsuperscript{-1} (soil). In addition, they demonstrated that although AFB1 appears to be transient in soil, it is apparently produced in surface soil in the presence of corn residues. This production was evidenced by \textit{A. flavus} CFU levels, detection of AFB1 in soil, and expression of genes related to aflatoxin biosynthesis. This is consistent with the results of this thesis, in which no aflatoxins were detected in soils from high-risk areas in Kenya, although samples were tested for soil fungi capable of producing aflatoxins (Chapter \ref{Chapter3}).


The factors and agricultural practices that influence the occurrence of aflatoxins in crops at the preharvest stage have already been studied (see Chapter \ref{Chapter3}). However, the influence of these factors on the occurrence of aflatoxins in soil remains largely unexplored. To bridge this knowledge gap, a large-scale field study was conducted in Chapter  \ref{Chapter3} within a high-risk model region for aflatoxin contamination in Sub-Saharan Africa, namely the Makueni region in Kenya. The objective of this study was to investigate the occurrence of aflatoxins in soils while identifying potential influences of agricultural practices, soil depth, and field location. Interestingly, no aflatoxins were detected in the soil samples. From these results, particularly the absence of aflatoxins in the soil of a model region at high risk for aflatoxin contamination, it could be concluded that aflatoxins are not present in soil at environmentally relevant levels. However, several factors challenge this conclusion. Notably, the absence of aflatoxins is likely due to degradation to undetectable levels during the 2.5-month transport (Chapter \ref{Chapter3}). Additionally, the occurrence of aflatoxins in the soil may be subject to a seasonal cycle. Given that aflatoxin-producing fungi were identified in the soil samples, it is plausible that \textit{in situ} production occurs during the early stages of crop cultivation, particularly when soil moisture recovers, leading to the germination of \textit{Aspergillus} sclerotia and spores, followed by the growth of the fungus \citep{accinelli2008aspergillus, elmholt2008mycotoxins}. Furthermore, heavily contaminated plant material, unsuitable for commercialization, is frequently incorporated into the soil post-harvest \citep{horn2003ecology, horn1995effect}, potentially representing a period of elevated aflatoxin concentration in the soil.


In conclusion, this work has revealed uncertainties regarding the extent of aflatoxin contamination in soil. Future research efforts should aim to investigate the temporal dynamics of aflatoxin occurrence in soil and explore the potential for in situ production by aflatoxin-producing fungi during the early stages of crop cultivation.

%----------------------------------------------------------------------------------------
\subsection{Dissipation of Aflatoxins in Soil Systems}
%----------------------------------------------------------------------------------------

In the context of assessing the environmental relevance of a substance, understanding its persistence in the environment is critical since the rate of dissipation has a central function in determining the duration and intensity of potential ecological effects.  Soil dissipation processes result from a combination of microbial, physical, and chemical factors. Previous literature, as reviewed in Chapter \ref{subchap:aflatoxins_in_soil}, suggested that aflatoxins in soil are rapidly degraded, with half-lives ranging from days to weeks, and that microbial degradation is the predominant dissipation process \citep{accinelli2008aspergillus, angle1980decomposition, angle1986aflatoxin}. In contrast, abiotic degradation processes in soil are generally considered negligible \citep{fouche2020aflatoxins}, an assertion that lacks empirical support, as only one experimental study has examined abiotic degradation in soil so far \citep{accinelli2008aspergillus}. However, given the short half-lives of aflatoxins under exposure to physical and chemical conditions such as UV light, organic acids, and ammonia, it is plausible that (photo)chemical degradation could contribute significantly to aflatoxin degradation in soil. Moreover, the interplay between microbial and (photo)chemical degradation processes in relation to available AFB1 concentration and soil physicochemical properties is still largely unexplored.

To address these knowledge gaps, Chapter \ref{Chapter4} presents a controlled laboratory experiment to systematically investigate the degradation of AFB1 in soil considering microbial, photochemical, and dark abiotic conditions in two different soil types (sandy loam and clay soil) and at varying initial AFB1 concentrations. Results showed AFB1 dissipation and AFB2a formation occurred in all soils and conditions. Notably, photochemical degradation emerged as a major degradation process, alongside the well documented predominance of microbial degradation. However, it should be noted that photodegradation is likely limited to AF contaminated material at the soil surface and in the topsoil due to the high light attenuation potential of the soil. Moreover, the determined half-lives of microbial degradation were considerably longer than previous studies, possibly due to drier conditions (40\% WHC) compared to earlier research with 80 - 100\% WHC \citep{accinelli2008aspergillus, angle1980decomposition, angle1986aflatoxin}. These findings suggest previous studies may have underestimated aflatoxin persistence in soil, particularly especially in drier conditions, such as those found in subtropical regions. In the sandy loam soil, higher initial AFB1 concentrations correlated with slower dissipation rates, likely due to toxic effects on microorganisms. This trend was absent in clay soil, probably due to reduced bioavailability by AFs sorption onto clay minerals. In all degradation scenarios, only AFB2a was detected as a transformation product, which is consistent with the findings of \citet{starr2017solvent}, who argued that the presence of the metabolites AFB2, AFG1, and AFG2 reported in previous studies \citep{angle1980decomposition, angle1986aflatoxin} resulted from misidentification, primarily attributable to the use of thin-layer chromatography. However, the amount of AFB2a formed did not account for the total dissipated AFB1. Mass balance analysis suggested a significant portion of dissipated AFB1 in a non-quantifiable fraction, whose exact nature remains unclear, whether it involves volatilization, mineralization to \ce{CO2}, bound residues, or incorporation into microbial biomass. Further investigations, such as radiotracer analysis, are needed to clarify this. 


In conclusion, my thesis underscored the significance of different degradation processes in determining the fate of aflatoxins in soil. For the first time, it was demonstrated that, alongside microbial degradation, (photo)chemical degradation can be a significant detoxification process, and that these processes are modulated by soil properties and initial aflatoxin concentration. These results contribute to the understanding of aflatoxins as micropollutants in the soil and highlight the role of soil properties in AFB1 degradation processes. Nevertheless, questions regarding the non-quantifiable contribution and the nonlinear effect of initial concentration on microbial degradation in clay soils remain unanswered, motivating further research.

%----------------------------------------------------------------------------------------
\subsection{Soil Environmental Implications of Aflatoxin Exposure}
%----------------------------------------------------------------------------------------

As outlined in Chapter 1.2, there is substantial evidence indicating that aflatoxins exert toxic effects on certain soil microorganisms. One plausible explanation for this phenomenon is that aflatoxins may be produced as a protective response to microbial competition or predation \citep{elmholt2008mycotoxins}. However, it should be noted that conflicting results exist in this regard, with some studies reporting toxic effects while others do not \citep{burmeister1966survey, arai1967antimicrobial, angle1981aflatoxin}. Critically, the majority of these effect studies were conducted under optimized \textit{in vitro} conditions, typically involving cultivation on agar media that do not consider soil as a natural environmental matrix \citep{drott2019fitness}. Moreover, these studies often focused solely on assessing the effects on microbial biomass, growth, and activity. This approach presents several limitations: (1) It excludes the influence of natural external factors to which these organisms may be exposed in the environment, factors that could significantly influence the magnitude and direction of the observed effects; (2) less than 1\% of the total microbiome can be cultured on agar media \citep{pham2012cultivation}, rendering the results non-representative of the entire microbiome; (3) it fails to assess the impact on the physiology and functionality of the microbiome, even though these aspects are crucially linked to soil functions. Current methods of disposing of crops contaminated with AFs, which often involve their incorporation into the soil, could result in elevated natural contamination levels and potential disruption of the ecological balance \citep{fouche2020aflatoxins}. This emphasizes the need for a comprehensive approach to gain a full understanding of the ecological function of AFs and to assess their potential impact on soil health. This should consider soil as a complex heterogeneous environmental matrix and examine microbial responses at different physiological levels.


To address this research gap, Chapter 5 presents a laboratory study that examined soil microbial responses to AF exposure across a range of environmentally relevant concentrations, focusing on multiple physiological response levels, including biomass, activity, carbon source utilization patterns and ecophysiological ratios, thereby considering soil as a complex heterogeneous environmental matrix.  Consistent with previous studies, it was shown that AFB1 at environmentally relevant concentrations had only minor and transient effects on soil microbial biomass and activity. Furthermore, the magnitude and direction of these observed effects depended on the soil type. Soil texture particularly affected AFB1 availability, which is consistent with observations on microbial and (photo)chemical degradation (Chapter \ref{Chapter4}). In clay soils, minor and transient stimulatory effects on catabolic functionality and microbial activity were observed, suggesting that AFB1 toxicity and availability were reduced by clay mineral-induced sorption, eventually leading to hormetic effects. This observation could also explain the nonlinear effect of initial concentration on microbial degradation in clay soils (Chapter \ref{Chapter4}). In contrast, sandy loam soils showed minor negative effects on catabolic functionality and microbial activity in response to AFB1 exposure, along with a slight increase in metabolic quotient.


In summary, it can be concluded on the basis of this thesis that aflatoxins do not pose a threat to the integrity of the soil microbiome and thus to soil health within the concentration range and time frame investigated. This is particularly true for clayey soils, where the toxicity of AFs is significantly reduced due to their strong binding to clay minerals. This relationship is consistent with research in various fields, including livestock, where clay minerals are used as binders in animal feed to reduce the uptake of aflatoxins by animals and thus mitigate potential harmful effects \citep{jaynes2007aflatoxin, wan2013toxicity, schell1993effects}. Therefore, these results highlight the critical role of considering soil structure, particularly clay content, in assessing the environmental impact of aflatoxins on the soil microbiome. Nevertheless, it is important to point out some limitations. No effects on community structure, particularly on the proportion of fungi in the biomass, were detected. However, changes in microbial composition cannot be excluded because the methodology used had limited taxonomic and physiological resolution. In addition, this study only examined the effects of a single AFB1 application event on German reference soils, which are assumed to have never been exposed to AFs. Soils in regions affected by aflatoxins, such as the (sub)tropical areas of Africa, are likely to be regularly contaminated with AF, which may lead to repeated exposure with unexplored longterm effects. In addition, these aflatoxin-impacted soils may face several stressors, including pesticides, fertilizer overuse, floods, and droughts. The interaction between these stressors and aflatoxins could change the magnitude and direction of the impact of aflatoxins on the soil microbiome and thus could impair soil health. Overall, this indicates that further research in the natural habitats of aflatoxin-producing fungi is needed to gain a more comprehensive understanding of the ecological importance of AFs to the soil microbiome and thus to soil health.

%========================================================================================
\section{Conclusion and Future Aspects}
%========================================================================================

The main objective of this dissertation project was to investigate the environmental relevance of aflatoxins in soil by scrutinizing the mechanisms and extent of aflatoxin occurrence in soil, the processes of their dissipation and their effects on the soil microbiome and associated soil functions, with regard to soil properties. Several methodological challenges that had previously hindered the investigation of the environmental relevance of aflatoxins in soil were successfully overcome. In particular, the development of a reliable and cost-effective analytical procedure has paved the way for aflatoxin research in the soil environment. Importantly, this method was designed with minimal cost and labor, making it applicable in resource-limited regions, particularly in subtropical areas where aflatoxin problems are widespread. A large-scale field trial was conducted with the aim of detecting aflatoxins in field soil and evaluating the influence of factors such as location, depth, soil properties and agricultural practices. The fact that no aflatoxins were detectable in this study highlighted that monitoring in the field remains challenging. These challenges include rapid degradation, spatial heterogeneity, and seasonality of aflatoxin occurrence, which must be considered in future field studies. Furthermore, this research has shown that aflatoxins undergo rapid dissipation in soil, highlighting the importance of abiotic degradation mechanisms, especially photolytic degradation, in the detoxification of aflatoxins in the soil. The influence of soil characteristics, particularly texture, on these processes has been underscored. Nevertheless, the causes of the dissipation of aflatoxins in soil remain uncertain and require further investigation in future studies. The study of the effects of aflatoxins on the soil microbiome and soil functions has shown that aflatoxins do not pose a significant threat to soil health, especially in clayey soils. 


However, important questions remain unanswered, highlighting the need for further research to gain a more complete understanding of the ecological significance of aflatoxins. Looking ahead, future research should focus on addressing the challenges of field monitoring of aflatoxins, elucidating the mechanisms underlying the dissipation processes of aflatoxins in the soil during microbial and (photo)chemical degradation scenarios, further investigating the ecological consequences of aflatoxins, especially in regions that are severely affected by aflatoxin issues, and exploring the complex interactions between aflatoxins and various environmental and anthropogenic stressors. By answering these questions, we can increase our knowledge of the environmental impact of aflatoxins on soil health and ultimately contribute to more effective strategies for managing aflatoxins in agriculture.
